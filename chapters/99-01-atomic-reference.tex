\chapter{Atomic Reference}\label{ch:atomic-reference}

Presented here is an atomically updatable reference to a Python object.
That is, a reference that can be safely updated in spite of concurrent changes and deallocations.

This was initially implemented in order to correctly handle the swapping of an \texttt{AtomicDict}'s internal metadata object (see~\S\ref{sec:migrations}), in a way that made use of Python's new QSBR scheme.
Its usefulness extends beyond the internal structures of \texttt{AtomicDict}: since the problem solved is very simple, its applications can be numerous.

The implementation described in this Chapter is publicly available at~\cite[src/cereggii/atomic\_ref.c]{cereggii}.


\section{Design}\label{sec:design}

The design of \texttt{AtomicRef} closely resembles the interface of Java's \texttt{AtomicReference}: an \texttt{AtomicRef} Python object is an object that stores a single pointer to another Python object.
(Regarding the slight name difference, the shortened \emph{ref} is more commonly used in Python, compared to \emph{reference}, both in reference counting C-APIs, and in other parts of the language.)
The choice of following the Java interface comes down to two considerations: programmers who are already accustomed to Java's interface need not learn a new one, and this (atomically updating a reference) being such a relatively simple problem, it would seem hard to design something completely different from the rest of the world.

The ``hard'' part of this design is to correctly use free-threading Python's QSBR scheme (see \S\ref{sec:qsbr} for an introduction).
Namely, in lines 23--25 of Listing~\ref{lst:atomic-ref} the read on line 24 is retried until the reference count increment succeeds on line 25.
This is where serious bugs may occur in a faulty implementation.
In fact, if incorrectly using QSBR, instead of failing a call to increment the reference count, it is possible that the memory pertaining to the fore-referenced object is freed, as that would be the effect of a concurrent \texttt{Decref} which sees the reference count reach 0.
The exhibited behavior would thus be a segmentation fault, a use-after-free, instead of an increase in the running time (due to contention).
A significantly worse possibility.

\begin{lstlisting}[
	caption={Pseudocode of \texttt{AtomicRef}'s core routines.},
    label={lst:atomic-ref},
    language=C,
    numbers=left,
	stepnumber=5,
	numberfirstline=true,
]
    typedef struct {
      PyObject_HEAD

      PyObject *reference;
    } AtomicRef;

    int AtomicRef_CompareAndSet(self, expected, new)
    {
      Incref(new);
      int success = CompareAndSet(&self->reference, expected, new);
      if (success) {
        Decref(expected);
        return 1;
      } else {
        Decref(new);
        return 0;
      }
    }

    PyObject *AtomicRef_Get(self)
    {
      PyObject *reference;
      do {
        reference = self->reference;
      } while (!TryIncref(reference));
      return reference;
    }
\end{lstlisting}


\subsection{Single-Field Struct}\label{subsec:single-field-struct}

An alternative design approach could have been not making \texttt{AtomicRef} a Python object.
Instead, it would be turned into a single-field C struct, essentially removing line 2 of Listing~\ref{lst:atomic-ref}.
This choice would slightly help the performance of \texttt{AtomicDict} as well, given that there are currently two indirections required to read the metadata object: one indirection to read \texttt{{dict->metadata}}, which is an \texttt{AtomicRef *}, and one indirection to read \texttt{{dict->metadata->reference}}, which is the actual \texttt{AtomicDict} metadata object.
Ipso facto, as an improvement for \texttt{AtomicDict} it would be very marginal.
Instead, if a user of this library would necessitate to manage several instances of \texttt{AtomicRef}, say, stored into a C array, access to each referenced object would be severely hindered by the double indirection.
Furthermore, \texttt{AtomicRef} not being a Python object would entail that it would not be possible to expose this presented functionality into Python code.

A possible approach to both gain further performance, and retain usability in Python code, would consist of using a single-field C struct for \texttt{AtomicRef}, as previously described, intended to be accessed from C shared library users, and also implementing a differently named Python type, say \texttt{AtomicRefObj}.
The new Python type would itself be the container for a single-field C struct \texttt{AtomicRef}, thus essentially retaining equal functionality as the one presented elsewhere in this Chapter.

This design proposal is inspired by the new \texttt{PyMutex} object of the original nogil fork, and of Python 3.13t.
The advantage of using a single-field struct is that the program using \texttt{PyMutex} has no need to be aware of its internal representations, however simple they may be.


\section{Issues with free-threading changes and proposed re-implementation}\label{sec:issues-with-free-threading-changes-and-proposed-re-implementation}

During an exchange with Gross, he pointed out that the method employed for correct concurrent handling of reference counts differed in Python 3.13t, compared to his original \texttt{nogil} fork.
Namely, that the \texttt{TryIncref} routine, shown above, would not behave in the same way, instead expecting the object to be in a \emph{weakrefs} state, as listed in~\cite[\S Biased Reference Counting]{pep703}.
In order to correctly implement \texttt{AtomicRef} for Python 3.13, it will therefore become necessary to enforce that the objects referenced to by an \texttt{AtomicRef} to be at least in the \emph{weakrefs} state.
The state transition needs to be carried out before a new reference is stored inside \texttt{AtomicRef}, i.e.\ before another thread may decrement the reference count, and then deallocate the object without using the QSBR slow path for deallocation.
Upon applying this modification, it is expected that the current implementation will restore correctness, without compromising the overall, presented design.
