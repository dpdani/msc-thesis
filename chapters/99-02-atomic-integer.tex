\chapter{Atomic Integer}\label{ch:atomic-int}

This chapter discusses the concept, rationale, and design of an atomic integer for Python, especially, but not exclusively, in the context of the free-threading changes.
The implementation discussed here is available in~\cite[src/cereggii/atomic\_int]{cereggii}.


\section[The necessity before and after the free-threading changes]{The necessity before and after the free-\\threading changes}\label{sec:the-necessity-before-and-after-the-free-threading-changes}

With the presence of the GIL, many concurrency problems that existed in Python code, and likely still exist, may never be observed.
An atomically updatable integer is thus a very simple abstraction to ensure that certain concurrency problems are handled correctly.

\begin{quote}
    [\ldots] There's currently no atomic int in Python and people go through all sorts of weird contortions to atomically count things (which often leads to broken code even with the GIL).\footnote{%
    This quote is from a comment by Gross in a public online discussion, and is accessible at \url{https://discuss.python.org/t/towards-a-python-util-concurrent/41325/8}, last accessed \today.
}
\end{quote}

Regardless of the Python build, a class of the likes of \texttt{AtomicInt} is still useful in Python's ecosystem.
It can additionally gain usefulness in Python free-threading build because of the performance improvements that can be gained with specific design choices, as described later in Section~\ref{subsec:atomic-int-handles}.


\section{Design}\label{sec:atomic-int-design}

The design of \texttt{AtomicInt} revolves around exposing common C atomic operations on 64-bit signed integers to Python code.
Therefore, most of the routines listed in file~\cite[src/cereggii/atomic\_int/atomic\_int.c]{cereggii} essentially create this mapping between C operations and Python APIs.

The APIs of \texttt{AtomicInt} closely resemble some of Java's \texttt{AtomicInteger} methods.\footnote{%
    See \url{https://docs.oracle.com/javase/8/docs/api///java/util/concurrent/atomic/AtomicInteger.html}, last accessed \today.
}
The \texttt{AtomicInt} has some easily expected methods, like \texttt{{get()}} and \texttt{{set(\ldots)}}, and also some borrowed from Java, and very similar to other C-level atomic operations, such as \texttt{getAndSet(\ldots)} and \texttt{{incrementAndGet()}}, among various others.

Especially useful are the so-called ``in-place'' Python methods, such as \texttt{my\_int += 1}, \texttt{my\_int *= 2}, etc.
These methods are very effective for atomic operations because, instead of requiring a separation of reads/writes in Python code, they describe the entire mutation that the program wants to see applied.
The operation can thus be efficiently retried within the same call to the relevant C routine, which performs the retry when the integer's value is contended, following the semantics of CAS atomic operations\@.

The program can thus not worry about the correct usage of atomic CAS primitives, when using in-place methods.\footnote{%
    Notice the similarity with \texttt{AtomicDict}'s proposed \texttt{aggregate} method, which permits to do essentially the same thing: fully describe a mutation, to be applied correctly by the internals of the routine.
}
Instead, a usage similar to the following would yield inconsistent results, assuming that the \texttt{my\_thread} function is effectively the code run by multiple threads:
\begin{lstlisting}[label={lst:atomic-int-bad}, language=Python]
    i = AtomicInt()
    ...
    def my_thread():
      for x in range(100):
        i = i + 1
\end{lstlisting}

What happens in this case is that the value that \texttt{i} is holding, is first retrieved, in a manner similar to calling \texttt{{i.get()}}, the sum is performed, and then the result is stored in \texttt{i}, similarly to calling \texttt{{i.set(\ldots)}}.
These three actions comprising a mutation that employs two distinct methods of \texttt{AtomicInt}, cannot be correctly executed without the protection of an external lock.
That is, the two methods being linearizable, does not entail that two successive calls to them are linearizable; in fact, the opposite is true.


\subsection{Handles}\label{subsec:atomic-int-handles}

It can easily be observed by profiling~\cite[examples/atomic\_int/counter.py]{cereggii}\footnote{%
    For instance, with a tool like Linux's \texttt{perf}, available online at \url{https://perf.wiki.kernel.org/index.php/Main_Page}, last accessed \today.
} that the running time is vastly dominated ($\approx$99\%) by concurrent reference counting and other Python thread-safety features, with the atomic increment being such a cheap operation in comparison.

An important observation is that those thread-safety features, and the concurrent reference counting thus required, are unnecessary for the correctness of \texttt{AtomicInt}, with its reference count being essentially modified only when the threads start using the object and then eventually releasing it, upon finishing their work.

Therefore, a simple, but very effective change was implemented.
The \texttt{AtomicInt} object does not necessitate being referenced directly by the thread that wants to mutate it, instead it is also possible to access it indirectly through a handle.
An \texttt{AtomicIntHandle} is an object that internally keeps a pointer to the \texttt{AtomicInt} object that it semantically references to.
Externally, it provides exactly the same APIs as \texttt{AtomicInt}, so that using or not the handle does not require substantial changes to a program's code.
If the handle is created by the only thread that uses it, as it is intended, then it enjoys the fast-path for concurrent reference counting implemented in free-threading Python.\footnote{%
    See~\S\ref{sec:concurrent-biased-reference-counting}.
}

The same reference counting operations, thus performed by the object's owner thread, would be substantially cheaper.
So much so, that the observed running time of~\cite[examples/atomic\_int/counter.py]{cereggii} drops by a staggering $\approx$95\%.

During an exchange with Sam Gross, the main author of the free-threading changes, he had indeed confirmed that the added burden of concurrent reference counting here exhibited is likely to remain in the free-threading build of Python, for the foreseeable future.
Furthermore, he was convinced that using this design of handle indirection can be effective in contrasting these performance bottlenecks.


\section{\texttt{AtomicShardedIntCounter}}\label{sec:atomicshardedintcounter}

Here is presented a possibility for future specialization of \texttt{AtomicInt}.
To further reduce contention, it is possible to restrict the usage of \texttt{AtomicInt} and then optimize it, based on the restricted semantics.

In fact, if the integer itself is used as a counter, a very common application, then it is possible for the $T$ threads accessing it to have $T$ distinct counters upon which to count in isolation with each other.
Since additions are commutative, the $T$ counters can be summed together lazily, only when the program wants to read the value of the counter, say by a \texttt{{get()}} operation.
The result might then be cached to avoid the repetition of work when the counter has not been changed.

In principle, this can be achieved in a manner similar to \texttt{AtomicDict}'s scheme for consistent size retrieval, described in~\S\ref{sec:consistent-size}.
The thread-local counters would be stored in a thread-local structure which also contains a lock.
Such lock would behave in the same manner as \texttt{AtomicDict}'s accessors locks.

The thread-local counters would need to be stored in distinct regions of memory, so as to ensure that they are not stored into the same cache line, exhibiting false sharing.
That is, the CPU cache coherency protocol would be obliged to keep the single cache line consistent among processors, resulting in no improvement, quite the opposite, over the current implementation of \texttt{AtomicInt}.
