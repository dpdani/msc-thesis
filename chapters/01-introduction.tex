\chapter{Introduction}\label{ch:introduction}


\section{Hash tables, concurrent hash tables, and their applications}\label{sec:hash-tables}

\cite{art-mp}

\section{Python's builtin hash table}\label{sec:dict-intro}

Python's hash table implementation, named \texttt{dict}, 

\cite{hettinger-dict}
\cite{dict-notes}
\cite{dict-comment-design}
\cite{dict-comment-hash}

\section{Free-threading Python}\label{sec:free-threading}

An account of the free-threading changes.

\cite{dabeaz-gil}
\cite{gross-doc}
\cite{mimalloc}

\paragraph{Thread identifiers.}
This is a unique thread identifier assigned by calling Python's \texttt{\_Py\_ThreadId} API\@.
Internally, it uses various hardware- and platform-dependent calls to generate the number.
For instance on x86-64 hardware running Linux, the identifier is stored in the FS register and is a pointer (i.e.\ a number) to the thread's \texttt{pthread} struct, used primarily for fast access to its Thread Local Storage (TLS).
Refer also to the Linux Kernel Documentation, \S29.8.1, Common FS and GS usage.
Available online at \url{https://www.kernel.org/doc/html/v6.9/arch/x86/x86_64/fsgs.html}, last accessed \today.

\paragraph{\texttt{PyMutex}.}

\paragraph{\texttt{PyEvent}.}

\subsection{Concurrent, biased reference counting}\label{subsec:concurrent-biased-reference-counting}

\cite{biased-refcounting}

\subsection{Deferred reference counting}\label{subsec:deferred-reference-counting}

\cite{deferred-refcounting}

\subsection{Quiescent State-Based Reclamation}\label{subsec:qsbr}

\cite{qsbr}

\subsection{Garbage collection}\label{subsec:python-gc}

\cite{pep703}

\subsection{Thread-safety of builtin data structures}\label{subsec:thread-safety-of-builtin-data-structures}

\cite[\S Container Thread-Safety]{pep703}
