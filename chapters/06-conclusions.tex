\chapter{Conclusions}\label{ch:conclusions}

We have shown that it is possible to implement an efficient concurrent hash table for CPython.
This is a very important building block for its evolving ecosystem, which is increasingly interested in parallelization.
To the authors' knowledge, this is the first naturally parallel hash table implemented for CPython.
In the previous Chapter we have experimentally shown the limitations of a hash table designed to be concurrent and compliant to CPython object protocol, and we have also shown that our designs to circumvent those limitations were effective.

These results were also presented at the 2024 Python Language Summit.
The handout of the presentation is included in Appendix~\ref{ch:language-summit}.
An article about the presentation and the discussion that followed is available online~\cite{python-summit-2024-free-threading-ecosystems}.
The article concludes with this note:
\begin{quote}
    Overall, the group seemed interested in Daniele's work on atomics but didn't seem willing to commit to exact answers for Python yet.
    It's clear that more experimentation will be needed in this area.
\end{quote}

What follows in this Chapter are interesting topics for further research and some closing remarks.

%\section{Further Research}\label{sec:further-research}

%Apart from the work already described in Chapter~\ref{ch:design-and-implementation}.


\section{Concurrent Garbage Collection}\label{sec:concurrent-garbage-collection}

In Chapter~\ref{ch:free-threading-python} we have seen numerous clever optimizations for making concurrent reference counting very efficient.
It is likely that for certain applications those optimizations are going to be fruitful.
In Section~\ref{subsec:measurements-overheads} we have shown that for some trivial applications those optimizations are not enough, and performance is clearly subject to concurrent reference counting.

A possibility that has not yet been explored for CPython is the design of a concurrent GC\@.
It could be beneficial for the performance of multithreaded applications that some objects live in a different space, one that is not subject to reference counting at all.
In that ``concurrent'' object-space a completely different mechanism could be employed for memory reclamation.
%For instance hazard-pointers, but several others are known in the literature.
Objects that live in that space could be shared more effectively between threads, without hindering the performance of the application.
The GC itself could also be designed not to be a stop-the-world GC, as is currently the case for CPython's, in order to permit fully lock-free data structure implementations.


\section{\texttt{AtomicDictHandle}}\label{sec:atomicdicthandle}

A significant portion of the measured time involving \texttt{AtomicDict}'s methods pertains to concurrent reference counting.
Such is because a lot of increments and decrements to the reference count are made when the CPython interpreter executes instructions that involve an instance of \texttt{AtomicDict}.
E.g.\ every time a lookup is performed, the Python interpreter increments and later decrements the reference count of the involved \texttt{AtomicDict} object.

Considering the necessity to use atomic operations to perform the counting (two in the above example), the amount of time spent in those routines is great ($\approx$15\%--25\% in the benchmarks of Chapter~\ref{ch:measurements}).
Instead of accessing the \texttt{AtomicDict} instance directly, though, it is possible to create a handle for it, like the one described in Section~\ref{subsec:atomic-int-handles}.
With the handle being a thread-local object, not intended to be shared with other threads, it can enjoy the fast-path for reference counting, thus reducing the time spent doing reference counting to a negligible amount.
This addition would not require a concurrent GC\@.

Since this would entail a reduction over the total amount of work that the CPU needs to perform, it is expected that an improvement of $\approx$20\% over the measurements of Chapter~\ref{ch:measurements} would be observed.


\section{Compatibility with Sub-Interpreters}\label{sec:compatibility-with-sub-interpreters}

Sub-interpreters are a fairly new technology, dating back to 2017, described in PEPs~554~and~734~\cite{pep554,pep734}.
Historically, since the GIL was deemed an intrinsic presence in the CPython interpreter, there have been various attempts at circumventing this limitation.
%Offloading the parallelizable computation to sub-processes, or custom C extensions are common workarounds to the GIL, as also noted by Gross in PEP~703~\cite{pep703}.
Simplifying a lot of the relevant history, sub-interpreters were also designed with this problem in mind, and were recently enhanced to be endowed with a per-interpreter GIL in PEP~684~\cite{pep684}.
Thus, several threads may have their own GIL, enabling them to run in parallel notwithstanding the continued presence of multiple global interpreter locks.
Given that the internal data structures of \texttt{AtomicDict} are Python objects, it cannot be used with sub-interpreters.
To the author, cross-compatibility of a data structure between free-threading \emph{and} sub-interpreters constitutes an interesting research area.

Sub-interpreter parallelization comes at the expense of not being able to easily share objects between interpreters, as also noted in~\cite[{\S}Per-Interpreter GIL]{pep703}.
Inheriting some of their design from Communicating Sequential Processes (CSP), the general idea is that the threads in sub-interpreters exchange information only through message-passing, rather than by directly sharing memory.
In fact, sharing \emph{objects} between interpreters is generally not allowed.
Nonetheless, there are mechanisms to share \emph{immutable} data between sub-interpreters.
In~\cite[\S Queue Objects]{pep734} is described a sub-interpreters-compatible implementation of a queue, that can be used to pass \emph{shareable} objects from one interpreter to another.
Integers, byte arrays, floats, and a queue instance in itself, are all shareable, among others.
Conceptually, an object qualifies as shareable as long as it can be entirely represented by an immutable array of bytes.

Crucially, a sub-interpreters queue is not a Python object, differently from \texttt{AtomicDict}.
It is instead allocated and freed without interacting with CPython's GC\@.
In order for a queue to be supported by sub-interpreters, its internal state cannot be stored into objects, because an interpreter accessing the queue would crash when trying to access an object not pertaining to its own object-space.
This also entails that there must exist an object \emph{representation} of a queue, which has to be created from the interpreter that desires to access the queue, and no other interpreter.

In fact, a sub-interpreters queue can be entirely represented by its ID number.
When a queue's memory is allocated outside CPython's object-space, the queue is inserted into a shared, global list of queues.
The ID of a queue is its position in the list of queues.
For a Python thread to access a queue, it is sufficient to know its ID, to then retrieve the queue from the list.
The object created to interact with a queue from Python code, i.e.\ the \emph{representation} of the queue, only knows the ID of the queue: it just holds an integer, and thus it can be entirely represented by an array of bytes.
Therefore, a queue can be considered \emph{shareable}, and it can also be enqueued into another queue, or even into itself.

This \emph{representational} behavior should hold some resemblance to the reader with the \texttt{AtomicDictHandle}, whose primary purpose has nothing to do with sub-interpreters, but that can nonetheless be used to hold an immutable, and thus potentially \emph{shareable}, representation of a hash table.
In fact, it can be entirely represented by an immutable pointer to the \texttt{AtomicDict} instance.
In order for the handle to work properly with sub-interpreters, an additional requirement is that the \texttt{AtomicDict}'s internal data structures should not be Python objects, putting further interest into implementing a different memory management scheme.

Supposing such is performed, more work needs to be carried out in order to ensure that the keys and values in \texttt{AtomicDict} are \emph{shareable}, by also making use of cross-interpreter object representations, in possibly a standardized way.
During an exchange with Eric Snow, author of the PEPs on sub-interpreters, the necessity for standardized, customizable ways to create cross-interpreter object representations emerged as one of Snow's planned works.
Snow intends to write a PEP proposing to create a new \texttt{\_\_xi\_\_} standard method for Python classes to return such immutable representations.

With these enhancements, albeit quite a substantial amount of work to implement them, it becomes possible to make use of the same \texttt{AtomicDict} implementation in both concurrency models: common multi-threading with free-threading, and isolated object-spaces multi-threading with sub-interpreters.


\section{Closing Remarks}\label{sec:closing-remarks}

Implementing a concurrent hash table was no trivial task.
Implementing one that adheres to CPython's protocols while maintaining good performance, has proven to be a terrifically difficult task.
I am glad I was finally able to reach this goal.

While I consider this to be quite an achievement, I don't consider this to be the final word on concurrent hash tables for the Python ecosystem.
I would in fact not want it to be that.
I rather hope this proves to be a starting point for concurrent data structures in this rich and evolving ecosystem.
