\chapter{Proposed Design and Implementation}\label{ch:design-and-implementation}

The implementation is available online at~\cite[src/cereggii/atomic\_dict]{cereggii}, and comprises 12 C source files, with Python bindings, and 29 test cases (written in Python code).

\cite{peniocereus-greggii}

$O(t)$

$\textrm{atw} \textrm{rgw} \textrm{rgr}$

The bindings to the Python runtime are considered an implementation detail, and are not discussed here.
They chiefly consist of mapping Python objects into C constructs, which is achieved with calls to Python's C-APIs.
Another feature of Python's C-APIs that has been extensively used is garbage collection, which is discussed in \S\ref{subsec:garbage-collection}.


\section{Design Overview}\label{sec:design-overview}


\subsection{Python's Dictionary}\label{subsec:python-dict}

A lot of the basis for \texttt{AtomicDict} has been inspired from Python's \texttt{dict}.
Namely, that the hash table itself is instead an index over another table called the data table.
In the latter is where the pointers to the keys and values objects reside.

This improves on both the storage space and the iteration speed.
In fact, if we call $F_r$ the fill ratio of the dictionary (i.e.\ the number of occupied slots over the total number of slots), there not need be $\sim(1 - F_r)$ amount of empty 16-bytes entries, for the unused slots in the hash table.
Instead, there will be $\sim(1 - F_r)$ empty $O(8)$-bytes slots in the index, while the keys and value pairs will be stored in a separate compact table.
That is, the data table will have a fill ratio of $\sim$100\%.
Thus, when iterating over the keys in the dictionary, the compact data table will serve as the basis for the iteration, instead of the sparse index.
Also, note that the size of index nodes will vary depending on the size of the key set: when the key set is small there is no need to index the data table with 64-bit numbers, instead the size of nodes is chosen dynamically, from a lower bound of 1-byte, when the cardinality of the key set is $\leq 64$.
As the set of keys becomes larger than the maximum allowed fill ratio,\footnote{%
	I.e. $2/3$, both in \texttt{dict} and \texttt{AtomicDict}.
} then the migration process is initiated, which does not usually touch the data table, but only the smaller index instead.
The data table is only modified when it needs to be reduced after a large number of deletes, which are supposedly modest in number.

As described in~\cite[\S13.1]{art-mp}, we should be inclined to think that the usage of the dictionary consists of 90\% lookups, 9\% insertions, and 1\% deletes.
In \S\ref{sec:dict-metrics} we confirm this simple rule-of-thumb.
We further expose a counter-intuitive notion: most lookups fail.
That is, most lookups return that the key searched for was not found in the hash table.\footnote{%
	See \S\ref{sec:dict-metrics} on \texttt{\_Py\_dict\_lookup} for the full details.
}

This may be explained by the ubiquitous presence of iterations, that is, of operations that return the entire key-value set.
Thus, it may be that in most cases, when a program looks up one specific key in the dictionary, it really isn't known whether the key is there or not.
And furthermore, instead of looking up all the individual keys which are known to be there, it may be that programs prefer to iterate over all keys.
This would result in individual lookups exhibiting a tendency to fail.

Based on this, we enhance the table hashing with the Robin Hood scheme, known in the literature~\cite{robin-hood,bolt}, which is especially effective in pruning searches for keys which are not in the table.
Our approach, detailed in the following \S\ref{subsec:lazy-robin-hood}, differs from~\cite{bolt} in that we don't necessarily constrain the dictionary to this scheme: we maintain the Robin Hood invariants, described later, as long as we can do so in a lock-free manner.
When the maintenance of those invariants requires an increasing number of atomic operations to be carried out in the index, we instead fall back to regular linear probing.
When doing so, before actually inserting a key that does not maintain the Robin Hood invariants, we mark the hash table as not anymore compact.
A compact operation can be requested by the user to recover the Robin Hood invariants.
This operation essentially consists of enlarging the hash table so that the invariants can be maintained.

We also optionally store tags in the nodes, further stealing some bits, in order to reduce the necessity of lookups into the data table.
The tag contains a portion of the hash of the key in the data table, when the tag does is not the same as the equivalent portion of the hash of the key looked up, the relevant entry in the data table is not visited.
This reduces the cost of having split index and data tables.

\paragraph{Python-generated hash values.}
The hash functions generated for built-in CPython data types enjoy the following properties:
\begin{enumerate}
	\item objects which compare equal have the same hash value;
	\item the computed hashes distribution is uniform, for a uniformly distributed set of values to be hashed;
	\item their running time complexity is $O(1)$;
	\item they don't necessitate holding locks, nor doing I/O, or whatever may be generally considered improper for a hash function\@.
\end{enumerate}
The reader may soon notice how these properties are generally true for all hash functions, with property (1) being fundamental for whatever hashing scheme to work, properties (3), and (4) being trivial for all hash functions known in the literature, and property (4) being what is usually the concern behind the implementation of hash functions.

With hash functions being so foundational to the implementation of hash tables, it is important to make one explicit remark.
There is no affordable guarantee that the hash functions we indirectly employ enjoy these properties too.
That is to say, a hash function written in Python code may perform whatever arbitrary operation.
For instance, it may very well acquire a lock, initiate a request to a remote machine on the network, read a large file from disk, or run an $O(n!)$ algorithm to produce the result.

Of course, such behaviors would be completely unexpected of a hash function.
Nevertheless, given their arbitrary nature, it is impossible to provide any sensible commentary on their enjoyment of the above properties.
We will therefore resort to assume that those properties hold, as we have no control over those functions.


\subsection{Concurrent Hash Table Design of Maier et al.}\label{subsec:maier}

Another substantial source of inspiration for \texttt{AtomicDict} is to be found in~\cite{maier}.
The core of the migration routine described later in~\S\ref{sec:migrations} is essentially the same as the one described in~\cite[\S5.3]{maier}.
In that work, Maier and his colleagues describe a table migration process that crucially eliminates most of the contention that can in principle arise from a migration.
In contrast with the standard literature, see for instance~\cite[Figure~13.30]{art-mp}, the migration process described does not necessitate the acquisition of one or multiple locks by a single thread that then takes care of performing the resizing.
In fact, in the entire~\cite[Chapter~13]{art-mp}, resizing (eq.\ migrating) is described only in terms of a ``stop-the-world'' operation.

That a fully lock-free migration process is possible has been shown by Hesselink et al.\ in~\cite[\S3.5]{hesselink}.
Notwithstanding the theoretical achievement, the practicality of such an approach can be subject to debate.
A migration process, as described in~\cite{hesselink}, comprises $\Omega(n)$ atomic memory writes (CAS).
Since an atomic write can be considered to be an order of magnitude more expensive than a normal write,\footnote{%
	For instance Intel SkyLake's \texttt{lock cmpxchg8b} has a measured latency of 23 clock cycles, while a \texttt{mov} instruction from a register to memory has a latency of 2 cycles.
	See~\cite[Intel Ice Lake and Tiger Lake]{x86-instruction-tables}.
} it is easy to see how the cost of migrations quickly becomes unsustainable, as noted in~\cite{maier} as well:

\begin{quote}
	While [\ldots] lock-free dynamic linear probing hash tables are possible, there is no result on their practical feasibility.
	Our focus is geared more toward engineering the fastest migration possible; therefore, we are fine with small amounts of locking, as long as it improves the overall performance.
\end{quote}

\texttt{AtomicDict} also follows the same philosophy.
It is clear that the cost of hardware primitives that implement compare-and-set memory semantics is very high when contrasted with the normal memory access primitives, thus they should be used sparingly.
When mutual exclusion helps to simplify the algorithm and its running time, it should be used.

We therefore maintain, like~\cite{maier}, that:

\begin{quote}
	[\ldots] The hash table works lock-free as long as no migration is triggered.
	% Once a migration is triggered, it is not lock-free but it happens asynchronously and it is hidden from any user of the hash table.
\end{quote}

Let us extend this concept by including other operations as well.
We will collectively refer to migrations and other kinds of operations that break the lock-freedom of the hash table as \emph{synchronous operations}, described later in~\S\ref{subsec:synchronous-operations}.
The essential characteristic is that their implementation may be greatly simplified by not allowing concurrent lock-free access; both from the perspective of computational costs, or maintenance costs.

Maier's hashmap implementation has an important difference when compared with Python's dictionary.
In the latter, the least-significant bits of the hash are used to determine the position in the linear array; that is, $d_0(x) = x \mod 2^s$, with $s$ being the logarithm base-2 of the dictionary size.
While in the former, the most-significant bits are used: $d_0(x) = x \gg (64 - s)$, where $\gg$ is a right-shift operation.

A reasoning as to why Python's hashmap chose this scheme can be found in~\cite{dict-comment-hash}:
\begin{quote}
	In a table of size $2^i$, taking the low-order i bits as the initial table index is extremely fast, and there
	are no collisions at all for dicts indexed by a contiguous range of ints.
	So this gives better-than-random behavior in common cases, and that's very desirable.
\end{quote}
On the other hand, the reason why Maier chose the right-shift based scheme is so that the relative position of two keys which are in the same probe doesn't change as much as in the alternative, when the array is resized.
That is, consider a key $x$ s.t. $h(x) \mathbin{\&} 2^{s + 1} = 0$, and a key $y$ s.t. $h(x) \mathbin{\&} 2^{s + 1} = 1$.
If they were colliding before resizing from $s$ to $s + 1$ (i.e. $h(x) \mathbin{\&} 2^s - 1 = h(y) \mathbin{\&} 2^s - 1$), they will not collide in the corresponding hashmap of size $s + 1$.
But this creates an unexpected problem.
It is trivial to see that the distance-0 after the migration $d_0'(x) = d_0(x)$.
The $d_0$ position of $y$ will have changed from $d_0(y) = h(y) \mathbin{\&} 2^s-1$ to $d_0'(y) = h(y) \mathbin{\&} 2^{s+1}-1$.
In other words, $d_0'(y) = d_0(y) + 2^s$.

The problem with this behavior is that the position in the new array cannot be safely determined by looking at the prior array, without reading the actual hashes of each key in a probe.

This problem is avoided by choosing the most-significant bits scheme: $d_0(y) = h(y) \gg (64 - s)$, and $d_0'(y) = h(y) \gg (64 - s - 1)$.
In other words, $d_0'(y) = d_0(y) \cdot 2 + 1$.
While $d_0(x) = h(x) \gg (64 - s)$, and $d_0'(x) = h(x) \gg (64 - s - 1) \Rightarrow d_0'(x) = d_0(x) \cdot 2$.

It follows that the nodes of a cluster in the prior array will ``stay close to each other'' in the new array.
This behavior enables avoiding the necessity to exercise care in the moving of nodes from one generation of the hashmap to the next, resulting in much less synchronization overhead.

Though, given this fundamental difference in picking the hash bits to determine the position in the hashmap, it was necessary to re-hash Python-generated hashes, or else the $d_0$ position of very many objects would be $0$.
This is done with a cheap CRC32 hardware instruction, that Maier also used in his implementation.
Thus, the required property of hashes being ``\emph{large} pseudo-random numbers,'' is respected.

As with regards to the different strategies that Maier et al.\ lay out to ``hide the migration from the underlying application,'' it was chosen to only implement the ``Recruiting User-Threads (u),'' strategy and not ``Using a Dedicated Thread Pool (p)''; and the ``Preventing Concurrent Updates to Ensure Consistency (s--semi-synchronized)'' strategy and not ``Marking Moved Elements for Consistency (a--asynchronous)''.
This decision came down to two considerations:
\begin{itemize}
	\item it would be surprising for the programmer to find out that a thread pool gets created at each instantiation of a hash table, and it would also not be much more performant according to~\cite[\S8.4, Using Dedicated Growing Threads]{maier}; and
	\item the semi-synchronized strategy seems easier to implement (and to explain), while also being more performant on average, according to Maier's own measurements.
\end{itemize}
Thus, in the following Section~\ref{sec:migrations}, we only present what Maier refers to as the \emph{usGrow} variant.


\subsection{Reservation Buffers}\label{subsec:reservation-buffers}

The usage of pages for the data table, implicitly creates zones of contention.
When threads want to add a new key into the hash table, the most-recently added page is the one in which the insertion is attempted.
So that effectively all threads are trying to contend one page of the data table.

The degree to which contention is exhibited depends on the strategy with which threads decide which free entry in the page they wish to take for the new item.
Consider a strategy in which the lowermost index is always chosen: every thread always tries to reserve the same entry, with only one thread succeeding at any given time.

A simple, yet much better, alternative is to treat the page itself as a sort of hash table.
That is, instead of choosing an entry based on the current availability, an entry is chosen based on the hash of the key.
Thus, with sufficiently uniformly distributed hash values, the contention is greatly reduced.

The reservation itself needs to be carried out with atomic operations, so that the cost of inserting one key is always at least two atomic operations: one write to the data table, and one write to the index.
In order to amortize the cost of having a data table separate from the actual hash table (the index), instead of reserving one entry at every insertion, we let threads reserve a configurable number of entries, four per default.\footnote{%
	The allowed values are 1, 2, 4, 8, 16, 32, or 64.
	With 64 being the size of an entire data table page.
}
When a thread has no available reservations, it resolves to finding a free entry based on the key's hash as described before, reserving four entries; instead when a thread has reservations at disposal, it directly writes into the free entries that it owns, using regular (non-atomic) writes.

In summary, the expected number of atomic writes per inserted key is $1 + 1/4 = 1.25$.


\subsection{Accessor Storage}\label{subsec:accessor-storage}

The reservation buffer, along with other things, is stored inside an accessor-local storage.
It contains:
\begin{enumerate}
	\item a mutex;
	\item a local view of the size of the dictionary;
	\item a flag indicating whether this accessor (thread) is participating in the current migration; and
	\item the reservation buffer.
\end{enumerate}

The mutex itself protects the contents of the accessor storage.
This may seem counter-intuitive, but it is actually very useful.
First of all, when a dictionary needs to be freed, all the allocated accessor storages need to be freed as well.
In order to do this, the accessor storages are kept in a list.
A thread freeing the dictionary traverses the list to free the accessor storages, and a thread accessing the dictionary for the first time, appends its newly created accessor storage to the list.

The mutex itself is thus generally held only by its accessor, which releases it at the end of any insert, or delete operation.

Furthermore, when a thread becomes the leader of a migration (detailed later), it may necessitate to modify the reservation buffers of the other threads, in case the data table is modified.
E.g.\ during a shrink migration, the data table is shrunk, as well as the index, in order to free unoccupied space.
That also entails that the entries in the reservation buffers need to be changed, because their location relative to the start of the data table has changed.


\subsection{Synchronous Operations}\label{subsec:synchronous-operations}

In abstract terms, the presence of the list of thread-local mutexes, described in the above Section, enables the creation of cuts in the execution of dictionary operations, inasmuch the acquisition of all the thread-local mutexes by one thread creates a distinction between the operations that happened before this circumstance and those that happened after it.

This crucial characteristic, which is required e.g.\ for the re-ordering of data table entries, enables also many more usages.
A few are described in the following sections.
In particular, it enables a very simple mechanism for ensuring that all accessors come to know the presence of a hash table migration, for establishing the correct size of the hash table, and for performing a sequentially consistent iteration of the items in the hash table.

All of these operations are called synchronous because they all share the common necessity to be carried out sequentially by one thread, or rather that at least one part of their execution needs to be performed sequentially.
For instance, the hash table migration enjoys the help of more than one thread, but requires a step in which the leader performs the necessary alterations to the data table before other threads can join in the migration.

The addition of this mechanism ensures that the dictionary presented here is capable both of performant lock-free operations, and of complex operations that require the exclusive access to the entire dictionary in order to be performed without a prohibitive number of expensive atomic memory writes.
The properties that a synchronous operation is sequential and ensured to be mutually exclusive w.r.t.\ all other threads, make it also very simple to be explained and understood, an important characteristic for collaborative development.

Do note that the presence of a synchronous operation does not impede the fulfillment of concurrent lookup operations, as also described in the relevant later sections.
The meaning is that for concurrent migrations, the lookup linearizes on the state of the hash table prior to the migration (the thread performing the lookup will not be participating in the migration process).
For other arbitrary synchronous operations, in principle this behavior may not be acceptable; nevertheless, for those considered so far, it is indeed acceptable.
This exemption of lookups from the participation in migrations follows along the lines of~\cite[\S5.3.2, Preventing Concurrent Updates to Ensure Consistency]{maier}, i.e.\ only mutations are prevented, read-only operations are allowed to proceed on the prior state of the hash table.


\subsection{Lazy Robin-Hood}\label{subsec:lazy-robin-hood}

The Robin Hood state for the hash table is essentially the distance each key has from its distance-0 position.
In other words, the number of collisions.

In order to keep track of this distance, we steal some bits from the nodes in the index.
According to~\cite[Corollary to Theorem~3]{robin-hood}, the expected number of collisions per probe for a full table is $\Theta(\log n)$, e.g.\ four for a table of size 64.
Such is a relatively low number, thus we can steal two bits, or in general $\log \log n$ bits, from the index nodes.

We do so, by following the values in Table~\ref{tab:robin-hood-nodes}.
As you can observe the proposed hash table has definite minimum and maximum sizes, respectively of $2^{6} = 64$ and $2^{56} \approxeq 7.2 \times 10^{16}$.

\begin{table}
	\centering\begin{tabular}{llll|llll}
		$\log n$ & Node & Distance & Tag & $\log n$ & Node & Distance & Tag\\
		\hline
		6 & 8 & 2 & 0 & 32 & 64 & 5 & 27 \\
		7 & 16 & 3 & 6 & 33 & 64 & 6 & 25 \\
		8 & 16 & 3 & 5 & 34 & 64 & 6 & 24 \\
		9 & 16 & 4 & 3 & 35 & 64 & 6 & 23 \\
		10 & 16 & 4 & 2 & 36 & 64 & 6 & 22 \\
		11 & 32 & 4 & 17 & 37 & 64 & 6 & 21 \\
		12 & 32 & 4 & 16 & 38 & 64 & 6 & 20 \\
		13 & 32 & 4 & 15 & 39 & 64 & 6 & 19 \\
		14 & 32 & 4 & 14 & 40 & 64 & 6 & 18 \\
		15 & 32 & 4 & 13 & 41 & 64 & 6 & 17 \\
		16 & 32 & 4 & 12 & 42 & 64 & 6 & 16 \\
		17 & 32 & 5 & 10 & 43 & 64 & 6 & 15 \\
		18 & 32 & 5 & 9 & 44 & 64 & 6 & 14 \\
		19 & 32 & 5 & 8 & 45 & 64 & 6 & 13 \\
		20 & 32 & 5 & 7 & 46 & 64 & 6 & 12 \\
		21 & 32 & 5 & 6 & 47 & 64 & 6 & 11 \\
		22 & 32 & 5 & 5 & 48 & 64 & 6 & 10 \\
		23 & 32 & 5 & 4 & 49 & 64 & 6 & 9 \\
		24 & 32 & 5 & 3 & 50 & 64 & 6 & 8 \\
		25 & 32 & 5 & 2 & 51 & 64 & 6 & 7 \\
		26 & 64 & 5 & 33 & 52 & 64 & 6 & 6 \\
		27 & 64 & 5 & 32 & 53 & 64 & 6 & 5 \\
		28 & 64 & 5 & 31 & 54 & 64 & 6 & 4 \\
		29 & 64 & 5 & 30 & 55 & 64 & 6 & 3 \\
		30 & 64 & 5 & 29 & 56 & 64 & 6 & 2 \\
		31 & 64 & 5 & 28 & \\
	\end{tabular}
	\caption{Sizes of internal fields of nodes in the index hash table. Let $n$ be the capacity of the table, and \emph{Node}, \emph{Distance}, and \emph{Tag} refer to their respective sizes, expressed in number of bits.}
	\label{tab:robin-hood-nodes}
\end{table}

Consider the general situation in which a new item is being inserted.
Paraphrasing from~\cite{robin-hood}:
\begin{quote}
	Suppose that element $A$ is at location $l$, at distance $d_a$ from its distance-0 slot, and $B$ is to be inserted (or moved).
	Suppose $B$ has been denied allocation of its first $d_b - 1$ choices and we are now considering $l$, its $d_b$th choice.
\end{quote}

Then, the Robin Hood invariant, or heuristic, maintains that:\footnote{
	As put by Celis et al. in~\cite{robin-hood}:
	\begin{quote}
		This procedure of ``taking from the rich and giving to the poor'' gives rise to the name Robin Hood hashing.
		Like Robin Hood, it does not change the average wealth (mean probe length), only its distribution.
		It is clear that the principal effect is to substantially reduce the variance.
	\end{quote}
}

\begin{tabular}{rp{8cm}}
	if $d_a \geq d_b$: & $A$ retains $l$, the insertion of $B$ continues by considering its $(d_b + 1)$st choice; \\
	if $d_b > d_a$: & $B$ displaces $A$ from $l$, the insertion procedure for $A$ is applied starting with its next $(d_a + 1)$st choice; \\
\end{tabular}

Comparing with~\cite{robin-hood}, we have slightly modified the invariant so as to favor avoiding moving elements on a tie.

The laziness in our approach is that if the \emph{aligned} size of the write required to safely insert the node into the index exceeds 128-bits, then we refrain from maintaining the Robin Hood invariants at all, for we would need to employ more than one atomic writes in order to store the newly inserted key.
This is because we cannot use the hardware~\cite[CMPXCHG--Compare and Exchange]{x86-64} instruction over more than 16-bytes.
The instruction further requires that the write be 16-byte aligned, or else a segmentation fault is encountered.\footnote{%
	Other faults may be encountered as well, depending on the architecture, for instance in aarch64 ISAs (which are not currently supported; see~\S\ref{subsec:compatibility-with-other-isas}) a bus fault is encountered.
}
Thus, e.g.\ if a new node is inserted into the index after a 16-byte boundary, the insertion cannot be performed with a normal (i.e.\ compact) node.

That is, when the distance cannot be stored in $O(\log \log n)$ space, we semantically say the distance is $\infty$ and store it as $\log \log n - 1$ (we may later refer to this as the maximum distance for a given size of the table, or simply as the maximum distance).
When a node has distance equals to the maximum distance, we say that node is a non-compact node.
We also employ non-compact nodes to denote tombstones.
As is well known, see for instance~\cite[\S6.4]{the-art-vol-2}, ``the obvious way of deleting records from a hash table doesn't work.''
Instead of simply removing a node, we swap it for a tombstone node, which is a non-compact node pointing to entry number 0 of the data table.
The data table maintains the invariant that the key and value fields of that entry are always \texttt{NULL}.\footnote{%
	This is achieved by requiring at initialization time that the thread creating the \texttt{AtomicDict} instance reserves entry number 0 for itself, and marks it as already occupied, without writing anything into it.
	Thus, all data tables have at least one unused entry.
	Furthermore, if the thread that created the hash table never writes into it, there will always be $|R|$ unused entries, with $|R|$ being the size of the reservation buffer.
	For more details on reservations see \S\ref{subsec:reservation-buffers}.
}
Operations that read the index, simply always skip non-compact nodes that point to entry 0.
Further details on tombstone nodes and deletes are given in \S\ref{sec:delete}.

When a new node could be inserted, respecting the requirements for our lazy robin hood, but it should be inserted at a distance greater then the maximum, we initiate a grow migration, as described in~\S\ref{sec:migrations}.
In effect, we have a local trigger for our migrations that doesn't require global knowledge of the table.
We can observe the local effect of collisions to determine that the table size is too small, and take action accordingly.
This is not the only trigger for migrations, for a complete list refer to the relevant section.
One may argue that a degenerate set of hashes can continuously trigger migrations.
While this is true in principle, do observe an implication of Table~\ref{tab:robin-hood-nodes}: starting from a table capacity of $2^9 = 512$, the maximum distance is 16, while the number of nodes in a 16-byte-aligned region of memory (a quad word), is 8.
Therefore, the described mechanism effectively cannot be triggered indefinitely.


\subsection{Garbage Collection}\label{subsec:garbage-collection}

As Micheal Maged put it in~\cite[\S2.3]{micheal-hash-tables}:
\begin{quote}
	[\ldots] The failure or delay of the garbage collector may prevent threads operating on lock-free objects from making progress indefinitely, thus violating lock-freedom.
\end{quote}

Such is undeniably correct, and therefore, the proposed implementation for \texttt{AtomicDict} cannot be considered lock-free, simply by virtue of the fact that its memory is managed through Python's garbage collector.
Notwithstanding, this was a deliberate implementation detail and not a foundational piece in the design of \texttt{AtomicDict}.
The fact that its internal data structures are traced with Python's GC is something that can be easily changed without compromising the design.
The choice was chiefly to simplify the implementation of \texttt{AtomicDict}.
Furthermore, given the fact that the hash table access was subject to the interpretation of Python code by the CPython interpreter, as that was the target for the presented work, there was no possibility of not having to deal with a stop-the-world GC\@.

Python code is not the only means through which \texttt{AtomicDict}'s APIs can be accessed.
The library also exposes C header files.\footnote{%
	Though, their standalone usage has not yet been explored.
}
For those use-cases that don't employ Python code, but rather directly call \texttt{AtomicDict}'s C functions, true lock-freedom can be of interest.

Changing the implementation to not be based on the provided GC, but instead using a dedicated memory management system, is entirely possible; albeit requiring a lot of additional work.
One way this can be achieved is by implementing what Micheal himself describes in~\cite{micheal-safe-reclamation} which employs so-called \emph{hazard pointers}; or any other suitable algorithm.

Serious effort was not put into looking at this possible implementation change, as of the time of writing.


\section{Lookup}\label{sec:lookup}

The proposed lookup design is what could probably be expected given the discussed design of the hash table.
The operation begins by computing the $d_0$ index slot, based on the pre-computed hash, by calling the \texttt{Distance0Of} function, which internally re-hashes the Python-generated hash,\footnote{%
	The hash computation is deliberately omitted here, because it essentially only entails making the correct call to the relevant Python C-API, and checking for possible errors.
} as described in~\ref{subsec:maier}, and then computes the slot position, based on the size of the hash table, listed in the \texttt{meta} object.

A fairly complete C-like pseudocode of its implementation is reported in Listing~\ref{lst:lookup}.

The algorithm then looks up the index for the searched element, starting from position $d_0$, until either:
\begin{enumerate}
	\item the item is found;
	\item an empty slot is found (the only probe in which the item could be found was looked at in its entirety);
	\item the hash table is in a compact state (i.e.\ no non-compact nodes are in the index) and the search can be pruned due to the Robin Hood invariant (i.e.\ the distance of the read node is greater than what the distance of the searched node would be if it was to be found); or
	\item the probe distance is equal to the index size.
\end{enumerate}

The searched key can be found if there is a node in the index whose tag corresponds to the key's hash, the node points to an entry other than 0, the entry's value is not \texttt{NULL}, the stored (Python-generated) hash is not different from the key's hash, and either the pointers of the searched key and the stored key are the same (identity), or the keys are semantically equal, based on the arbitrary logic with which those two Python objects may be deemed equal.

It can be seen that in lines 38--55, care is taken in order to delay the call to the comparison function as much as possible.
This is because comparison between arbitrary Python objects can be very computationally expensive.
Since the involved object's types may be implemented in Python code, it often requires interpreting an arbitrary Python function.
Such function does not have any restrictions on its computational complexity, and even if it was enforced to be an $O(1)$ operation (as it would be normally expected), its constant factor would still be very high, due to interpretation overhead.
The comparison itself may also raise an exception, either if the two objects' types are not comparable, or if the arbitrary Python code that was executed to compare the two objects raised an exception for any reason.
For the sake of discussing the running time complexity of our routine, we will make the (very possibly wrong) assumption that the cost of calling the Python C-API is always $O(1)$, regardless of the types of the two objects.\footnote{%
	Comparisons in Python are notoriously costly.
	As is well known, Timsort was implemented specifically for Python, with the explicit goal of minimizing the amount of comparisons as much as possible.

	Also note that in Listing~\ref{lst:lookup}, the called pseudo-function is not the actual function, but an abstraction that eases discussion, without really impacting it.
}

Do note that in the \texttt{ReadEntry} sub-routine the key is always read before the value.
This avoids the problem of torn-reads, exactly as described in~\cite[\S4, Lookup]{maier}.

\lstinputlisting[
	numbers=left,
	stepnumber=5,
	numberfirstline=true,
	language={C},
	caption={Lookup Operation},
	label={lst:lookup}
]{listings/lookup.c}

\subsection{Linearization}\label{subsec:lookup-linearization}

Towards the end of Listing~\ref{lst:lookup}, at lines 68--94, is where the response is written into an externally allocated C \texttt{struct}.
See how on line 69 the eventuality that the compact state of the hash table has changed is taken into account.
Suppose the check on line 69 didn't exist, and consider two threads $t_1$ and $t_2$ executing respectively a lookup for key $k$ and an insert for the same key.
There can exist an execution s.t.:
\begin{enumerate}
	\item $t_1$ runs first and reaches line 8, reading that \texttt{is\_compact} $= \top$;
	\item $t_2$ runs from start to end, turning \texttt{{meta->is\_compact}} $= \bot$, and inserting $k$ with a non-compact node; and
	\item $t_1$ runs again, not finding the node pertaining to $k$, thus returning a lookup failure.
\end{enumerate}

Instead of this evidently faulty behavior, the check on line 69 makes the lookup restart from the beginning, so that item (3) above would be substituted ``$t_1$ at first not finding $k$, sees that a non-compact insertion could have linearized, undermining its assumption that the Robin Hood invariant holds, and thus restarts from line 7, eventually finding $k$.''

Another similar check can be found on lines 72 and 79, where concurrent migrations are considered.
It is important to note that lookups are permitted to run concurrently with migrations, as they are read-only operations.
When a concurrent migration from the generation of \texttt{meta} to a newer generation has been linearized (setting this metadata's \texttt{{migration\_done}} flag to $\top$)\footnote{%
	This is actually implemented with an atomic event, so that threads that are waiting for the migration to finish are awaken.
} it is possible that a subsequent insert (potentially the one that initiated the migration) for the same key as the lookup operation linearizes before the lookup operation completes, in a manner similar to the previously discussed compactness of the hash table.
In this case, the $d_0$ position of the key must also be recomputed.

Other than the two cases described, the successful lookup linearizes on line 41, where the key is read.
Failed lookups, i.e.\ lookups that don't find the searched key, linearize on line 15, where a node is read from the index.


\paragraph{Lock-freedom.}

It is trivial to see that no thread's arbitrary delay can induce this lookup to also be subjected to an arbitrary delay, insofar as the code presented is concerned.

It is true that there is one aspect that evades the control of the author, namely the equality comparison.
It can indeed be that the arbitrary code that is run also comprises a lock acquisition.
Lock-freedom of that relevant code can in no way be evinced, nor enforced.
Therefore, we can only assume\footnote{%
	Or rather, hope.
} that the arbitrary code is sane and does not exhibit such surprising behavior.


\section{Insert or Update}\label{sec:insert-or-update}

The \texttt{ExpectedInsertOrUpdate} routine shown in Listing~\ref{lst:insert} departs from Maier's design of \cite[Algorithm~1]{maier} in that it doesn't expect the user to provide an update function, but rather mimics more the compare and set routines of other atomic data structures: it takes as input the expected and desired values, and it returns the value that was stored before the update was performed (or a special \texttt{{NOT\_FOUND}} object).
If the expected value was not the stored value (or the compare and set low-level call failed and the newly read value then differed from the expected one), then a special \texttt{{EXPECTATION\_FAILED}} object is returned.
A routine that more closely resembles Maier's Algorithm~1, based on the one discussed here, is described later in~\S\ref{sec:aggregate}.

There's also another special value called \texttt{ANY} that can be used as the expected value to signify that whatever the current value is, even an absent value, it should be replaced with the desired value.
Returning the previous value, in conjunction with \texttt{ANY} as the expected value, also serves to reuse the same routine to behave like a swap, rather than a CAS\@.

The three special objects \texttt{{ANY}}, \texttt{{NOT\_FOUND}}, and \texttt{{EXPECTATION\_FAILED}} cannot be used as keys or values in \texttt{AtomicDict}, so as to guarantee their semantics are consistent (cf.\texttt{\ {EXPECTATION\_FAILED}} being a value returned by this routine).
This is enforced, but not otherwise shown here.

If the expected value is either \texttt{{ANY}} or \texttt{{NOT\_FOUND}}, there is an insertion fast-path used in which the $d_0$ element in the index is CASed with the entry reserved without going through the currently stored data in the index.
If this CAS succeeds, then the operation is completed.
If it doesn't the general code path is followed.
Notice that this doesn't compromise correctness: if the $d_0$ node was in fact empty, it means that the key wasn't in the hash table.

Here, it is worth mentioning that this routine is not directly exposed, and is instead considered internal.
Instead of calling this function directly, a user of \texttt{AtomicDict} is expected to call different Python idioms, which all internally call \texttt{ExpectedInsertOrUpdate}; namely:

\begin{enumerate}
	\item \texttt{{d[k] = v}}, eq. \texttt{{ExpectedInsertOrUpdate(\ldots, k, \ldots, ANY, v, \ldots)}};
	\item \texttt{{d.compare\_and\_set(k, exp, des)}}, eq. \texttt{{ExpectedInsertOrUpdate(\ldots, k, \ldots, exp, des, \ldots)}}, which raises an exception when the expected value was not the stored value;
	\item \texttt{{d.swap(k, v)}}, eq. \texttt{{ExpectedInsertOrUpdate(\ldots, k, \ldots, ANY, v, \ldots)}}, which is similar to (1) except it also exposes the previous value;\footnote{Not yet implemented.}
\end{enumerate}

In regards to (2) above, returning a boolean value is generally considered the standard.
Such is what the hardware primitive essentially does, and also what other languages such as Java do.
Nevertheless, in order to favor the usage of the Aggregate routine of~\S\ref{sec:aggregate}, which correctly implements a generic mutation to a value, it was chosen to expose this behavior.
That is, simpler usages, such as incrementing a counter, or any other ``aggregations,'' should instead employ the more appropriate Aggregate routine.
This \texttt{compare\_and\_set} method is thus intended to be used in scenarios where the expectation is ``taken for granted,'' and would otherwise require debugging if unmet; that is, where the failure of the expectation has more severe impacts than mere contentious access.
Do note that this design does not pertain any correctness or performance goal; it, instead, has to do with the programmer's ergonomics in the usage of this lower-level routine.
Not every programmer possesses thorough knowledge of concurrency issues, and an interface that alleviates some of those complexities is what the author strives to achieve.
Whether or not this is actually simpler to use remains to be seen.

The \texttt{ExpectedInsertOrUpdateCloseToDistance0} sub-routine is omitted for brevity.
What it does can be described with the following steps:
\begin{enumerate}
	\item consider the nodes in the 16-byte-aligned region of memory the $d_0$ slot is part of, in sequence, starting from $d_0$:
	\begin{enumerate}
		\item if an empty slot is found either set \texttt{expectation} $=\bot$, if \texttt{expected} is neither \texttt{ANY} nor \texttt{{NOT\_FOUND}}, or attempt to insert the node in keeping with the Robin Hood invariant;
		\item if a non-empty slot is found, check the tag, and possibly call the \texttt{ExpectedUpdateEntry} routine;
	\end{enumerate}
	\item set the \texttt{{must\_grow}} flag if the node would be inserted at a distance greater than the current maximum distance (and avoid modifying the index);
	\item set the \texttt{done} flag accordingly.
\end{enumerate}

The \texttt{ExpectedUpdateEntry} sub-routine, which is also employed as a sub-routine of \texttt{ExpectedInsertOrUpdateCloseToDistance0}, is also omitted for brevity, its implementation resembles the \texttt{{check\_entry}} part of the \texttt{Lookup} routine (Listing~\ref{lst:lookup}, lines 38--65), but also performing an update, possibly signaling that the expectation failed.


\lstinputlisting[
	numbers=left,
	stepnumber=5,
	numberfirstline=true,
	language={C},
	caption={ExpectedInsertOrUpdate Operation},
	label={lst:insert}
]{listings/insert.c}

\subsection{Linearization}\label{subsec:insert-linearization}

\subsection{Lock-freedom}\label{subsec:insert-lock-freedom}


\section{Delete}\label{sec:delete}

\lstinputlisting[
	numbers=left,
	stepnumber=5,
	numberfirstline=true,
	language={C},
	caption={Delete Operation},
	label={lst:delete}
]{listings/delete.c}

\subsection{Linearization}\label{subsec:delete-linearization}

\subsection{Lock-freedom}\label{subsec:delete-lock-freedom}


\section{Migrations: growth, shrinking, and compaction}\label{sec:migrations}

\subsection{Linearization}\label{subsec:migration-linearization}

\subsection{Non-lock-freedom}\label{subsec:migration-lock-freedom}


\section{Approximate size}\label{sec:approximate-size}

An approximation of the number of elements stored in the hash table is inferred by the number of pages present.
Three separate, atomic counters are kept:

\begin{itemize}
	\item the greatest allocated page $P_A$;\footnote{%
		This counter is also used to keep track of the used portion of the pages array, and is the one considered when a thread makes a new reservation into the data table.
	}
	\item the greatest deleted page $P_D$; and
	\item the greatest refilled page $P_R$.
\end{itemize}

The three counters serve to provide a rough estimate on the number of elements currently stored: as new elements are inserted new data table pages need to be allocated, and when numerous elements are deleted they get defragmented into their own pages, before the pages get eventually freed, or be refilled.
Note that initially, when there are yet to be inserted any elements, their respective values will be 0, -1, and -1.

So an upper bound for the number of elements contained in the dictionary is:
\[
(P_A - P_D + P_R) \times |P|
\]

The bound is then also refined by visiting the $P_A$ and $P_R$ pages, decrementing it of the amount of elements that entries that weren't filled in those pages.
There is also a lower bound, which is computed by decrementing the upper bound by the unused reservations, accounted by traversing the accessors' storages.
Therefore, the running time complexity of this operation is $\Theta(t)$.

The mean (floor division by 2) of the lower and upper bounds is then reported to the user upon request.

\paragraph{Linearization and lock-freedom.}
This is trivially a lock-free algorithm.
The linearization here depends on the three reads required for $P_A$, $P_D$, and $P_R$, as well as the $\Theta(t)$ reads required to traverse the accessors' storages.

???

\section{Consistent size}\label{sec:consistent-size}

Given that the linearization of the above approximation of the size of the hash table is sub-optimal at best, it was decided that there would also be a definite, sequentially consistent size retrieval of the hash table.
This is achieved by using thread-local size counters: each thread keeps track of how many items it has inserted, incrementing its local counter, and how many items it has deleted, decrementing its local counter.
(Thus, it may be that one thread's local counter is negative.)
The thread-local counters are kept in each thread's accessor's storage, and are recorded by using regular reads and writes: they are protected by the accessor's lock.

Upon request, a Synchronous Operation (\S\ref{subsec:synchronous-operations}) begins in order to correctly sum the thread-local counters.
The counters are reset to 0 before the Synchronous Operation ends.
The result is cached into the \texttt{AtomicDict} and a dirty flag is kept in order to possibly avoid restarting a Synchronous Operation at every request, if no insertions or deletions marked the cached value as dirty.
That cached value is stored with a Python integer.
Upon a new request, if the dirty flag is set, the sum of the thread-local counters, plus the cached value is computed and stored.\footnote{%
	Such happens always, but initially the counter is set to 0.
}

Note that it is much faster to keep thread-local counters and then acquire locks to read them, rather than to keep a shared counter.
Using a Synchronous Operation for this also implies that its semantics are trivially sequentially consistent.
It can be argued that to retrieve this consistent size, there is a missed opportunity cost related to the exclusion of all other threads from using the hash table.
While true in principle, it should be noted that this is an $\Theta(t)$ operation, thus independent of the table size, and generally quite fast.
For these reasons, this is the routine that is exposed by default.
If a program cannot withstand the added cost of mutual exclusion (perhaps because it frequently asks for the size of the hash table, whilst this is being concurrently mutated), it is still possible to access the approximate size described in the above Section, which is quite accurate and lock-free, whilst also being a fast operation.
Additionally, under the situation of high-frequency concurrent mutations, it may make more sense to retrieve an approximation of the size rather than a sequentially consistent size.
The sequentially consistent size computed during a call to this routine, may be invalidated immediately upon returning its output.

Notwithstanding the limitation on the number of nodes that can be present into the presented hash table at any given time ($2^{56}$), as prescribed by Table~\ref{tab:robin-hood-nodes}, a counter of 64-bits is not necessarily sufficient.
Consider the situation in two threads concurrently insert and delete $2^{64}$ items, so that one thread only performs insertions, and the other only performs deletions (or any other equivalent situation).
Care then needs to be taken in order to avoid running into integer overflows or underflows, a notorious presence in C code.
Before such an event occurs, the thread that detects it (its local counter is \texttt{MAX\_INT64} during an insert, or \texttt{MIN\_INT64} during a delete), the thread repeatedly tries to add (CAS) its local counter to the \texttt{AtomicDict}'s cached value, while holding its local accessor's storage mutex.
This way, no Synchronous Operation can begin, and it is safe to add the protected value, before resetting it to 0.
If another thread finds itself in the same condition, the competing threads will serialize based on the CAS to the cached counter.

\paragraph{Linearization and non-lock-freedom.}
With this being a Synchronous Operation, it is trivial to see that it is not lock-free, and its linearization follows that of other Synchronous Operations.

\section{Batch lookup}\label{sec:batch-lookup}

Inspired by the design of DRAM-HiT, this method was added so as to amortize the cost of multiple memory accesses when a program wants to read a number of keys from the hash table.
It deviates from~\cite{dramhit} in that it does not require the user to continuously poll the table for results.
Instead, a batch of keys can be submitted for lookup, in one invocation.

An example call:

\begin{lstlisting}[language=Python]
foo = AtomicDict({'a': 1, 'b': 2, 'c': 3})
result = foo.batch_getitem({
  'a': None,
  'b': None,
  'f': None,
})
assert result == {
  'a': 1,
  'b': 2,
  'f': cereggii.NOT_FOUND,
}
\end{lstlisting}

The \texttt{cereggii.NOT\_FOUND} object is a special, global object that cannot be inserted into \texttt{AtomicDict}.
The fact that it is in the output means, as its name implies, that the key was not in the dictionary.

The operation involves the prefetching of each distance-0 location of the keys in the batch.
This results in an amortization of the cost of individual memory accesses: if the number of prefetches issued is sufficiently large (how large depends on the hardware), then the processor will be stalled waiting for the first read, while the subsequent ones will be read from the cache without incurring in the cost of main memory access.
This behavior is cheaper-by-comparison: it is cheaper to prefetch several keys, possibly blocking at the first one read, than to block at every read.

NUMA architectures will enjoy this mechanism particularly: when prefetches are issued at remote memory addresses, the comparative cost of the amortized reads is much less than the sum of the costs of each individual read.

In talking about NUMA architectures, it is generally also possible to argue that the cost of a remote access may be hidden by the possibly many local memory accesses that precede it.
Such is not our case: the index, which is allocated as a contiguous array, will reside entirely in one NUMA node.
Thus, a thread accessing the index will either only issue operations to remote memory, or to local memory.\footnote{%
	A further endeavor may comprise the partitioning of a hash table into several shards, which in turn are managed by one or multiple threads associated to a single NUMA node, such that any given key may only reside in one shard (e.g. by using the least-significant bits of the hash to associate a key with a shard).
	
	Mimicking the design of DRAM-HiT, the user thread would submit requests to a message queue which is consumed by the threads associated with a shard.
	The design need not be asynchronous as in \cite{dramhit} to gain the speed-up of stronger memory associativity. XXX check this
}
NUMA support is thus limited: the presence of a single shared index makes it impossible to efficiently make use of the memory partitioning.

Further work may be carried out so as to provide an \texttt{AtomicShardedDict} class which creates its own pool of threads, bound to individual NUMA nodes, and affords access to the hash table shards to those threads only.
This is resembling of the design of~\cite{dramhit}.
It can be implemented by making use of the current implementation for \texttt{AtomicDict}, such that one shard of \texttt{AtomicShardedDict} is an instance of \texttt{AtomicDict}.
Individual keys may be uniquely assigned to a shard based on their hash.

The batch is subdivided into chunks of a configurable size.
This is intended to prevent the over-fetching of memory.
That is, the cache of an individual processor may become overly filled with the prefetched portions of the index, and the hardware will proceed to evict lines of cache before they are actually read by the lookups that requested them.

\subsection{Linearization}\label{subsec:batch-lookup-linearization}

The linearization of this method is intended as the individual linearizations of the distinct lookup operations.
Each distinct lookup is linearizable, although the overall invocation is not.
Therefore, it is possible for a result of this method to not be sequentially consistent.


\subsection{Lock-freedom}\label{subsec:batch-lookup-lock-freedom}


\section{Aggregate}\label{sec:aggregate}

A notorious behavior of using CAS semantics is to repeatedly make the call to the CAS routine in order to cope with concurrent mutations to the contested area of memory.
We are especially interested into efficiently and ergonomically handling the mutation of the value of a given key, for such is an important use-case for hash tables in general.

We here propose, without an accompanying implementation, an interface that is believed to satisfy the both the efficiency and ergonomics goals, which is also very similar to~\cite[Algorithm~1]{maier}.
Instead of providing an expected and a desired value, the programmer specifies the desired mutation.
Consider the following piece of Python code.
\begin{lstlisting}[label={lst:aggregate-usage}, language=Python]
	d = AtomicDict({...})

	d.aggregate(key="spam",
	  mutation=lambda current_value:
	    1 if current_value is NOT_FOUND else current_value + 1
	)
\end{lstlisting}
Do note the \texttt{lambda} expression on line 4, which is a Python idiom to define an anonymous function, within an expression.

A programmer using this method should be instructed to write a function that handles both the cases of the key being present or not, and that there should be no assumptions as to the number of times this function is called before its described mutation is actually applied to the value.

That function is thus called, possibly multiple times by a routine that behaves like the following:
\begin{lstlisting}[label={lst:aggregate}, language=Python]
	expected = d.get(key, cereggii.NOT_FOUND)
	while True:
	  try:
	    d.compare_and_set(key, expected, mutation(expected))
	  except cereggii.ExpectationFailed:
	    expected = d.get(key, cereggii.NOT_FOUND)
	  else:
	    break
\end{lstlisting}
Such routine would likely be implemented in C code, for the sake of uniformity with the rest of the implementation of \texttt{AtomicDict}.

The very specific method of calling a CAS method in the above Listing~\ref{sec:aggregate} is very commonly known in the literature, albeit certainly more common to be written in a form where the CAS operation returns a boolean, rather than raising an exception.\footnote{%
	See also the relevant discussion in~\S\ref{sec:insert-or-update}.
}
The fact that this is well known in the literature, though, does not directly imply that it is also well known in the literacy of Python programmers.
For many programmers who are not accustomed to the domain of concurrency, the above Listing should probably be explained and broken down, before its semantics are fully appreciated.
To provide a transparent and standardized way to access such shared knowledge is therefore considered fruitful.

Do note that the atomicity of this \texttt{aggregate} method is violated if within the provided mutation function, a lookup into \texttt{d} is performed.
Or in other words, this is not a suitable design for a mutation that requires a sequentially consistent view of more than one key in the dictionary.

\subsection{Reduce}\label{subsec:reduce}

In order to provide further performance improvements, another method could be added, already otherwise known in computing as \emph{reduce}.
Such is a foundational concept of many computing models, most notably those based around the concept of map-reduce like for instance Hadoop.
As such, it could easily be provided based on the above \texttt{aggregate} method, and further enhanced to reduce contentious access to the dictionary.
Its usage would be something resembling the following lines of code:
\begin{lstlisting}[label={lst:reduce-usage}, language=Python]
	d = AtomicDict({...})

	batch = [
	  (key_1, value_1),
	  ...
	  (key_n, value_n),
	]

	d.reduce(iterator=batch,
	  mutation=lambda key, current_value, new_value:
	    new_value if current_value is NOT_FOUND
	    else current_value + new_value
	)
\end{lstlisting}

The mutation function here is made more complex because it needs to be also independent of the key and value: the same function needs to be applied for every key-value pair in \texttt{iterator}.
The parameter \texttt{key} is added so as to satisfy the possible cases in which the computation is dependent on the key itself.\footnote{%
	It is immediatly clear for a customary Python programmer that the \texttt{lambda} expression could be exchanged for a full non-inline function definition using \texttt{def}.
	Such is the expected usage for more complex mutations than the trivila one shown here.
	Nevertheless, it is worth pointing this out for the less Python-expert reader.
}

Contention can be reduced by first accumulating an intermediate result into a thread-local hash table, and then applying the \texttt{aggregate} method for each item in the local hash table.
This reduces contention because the repetition of keys in the passed iterator is not made visible to other threads, but only to the thread-local hash table.

Since this thread-local hash table certainly requires allocation of memory, its size could also be bound.
Upon reaching the bound the accumulated intermediate result is applied to the \texttt{AtomicDict}, and then the operation is repeated until the exhaustion of \texttt{iterator}.


\paragraph{Linearization and lock-freedom.}
The linearization and lock-freedom of \texttt{aggregate} and \texttt{reduce} is chiefly dependent on the respective properties of the \texttt{ExpectedInsertOrUpdate} routine that is being called in order to implement them.
Thus, refer to~\S\ref{subsec:insert-linearization} and \S\ref{subsec:insert-lock-freedom} for further discussion.


\section{Iterations}\label{sec:partitioned-iterations}

\subsection{Iteration Partitioning}\label{subsec:iteration-partitioning}

\subsection{Linearization}\label{subsec:iter-linearization}

\subsection{Lock-freedom}

\section{Proposal for a consistent iteration}\label{sec:consistent-iteration}

When a thread iterates over the hash table with the iterator described above in~\S\ref{sec:partitioned-iterations}, it may encounter sequentially inconsistent behaviors, if other threads are concurrently mutating the table.
For instance:
\begin{enumerate}
	\item the same key is emitted more than once; or
	\item an update $u_1$ that happened strictly before another $u_2$ is not seen, while the latter is.\footnote{%
		Such is the case when two updates are executed in succession by the same thread.
		Thus, it is definitely known that $u_1 \rightarrow u_2$.
	}
\end{enumerate}

These may be very surprising behaviors for a programmer trying to debug an incorrect execution of an iterating thread.
The necessity for a sequentially consistent iterator lies therein.

Those behaviors may be exhibited in the following circumstances, respectively:
\begin{enumerate}
	\item the iterating thread $t_l$ visits the data table entry $e$ in which some key $k$ is stored $\rightarrow$ another thread $t_d$ deletes $k$ $\rightarrow$ another thread $t_i$ re-inserts $k$ at a location $e'$ in the data table s.t.\ $e < e'$ $\rightarrow$ $t_l$ visits $e'$ in which $k$ is stored.\footnote{%
		It may also be that $t_d \equiv t_i$.
	}
	\item the iterating thread $t_l$ visits the data table at location $e$ where the key $k_1$ is stored $\rightarrow$ update $u_1$ to key $k_1$ is performed by another thread $t_{u_1}$ $\rightarrow$ update $u_2$ to key $k_2$, stored at location $e'$ s.t. $e < e'$ is performed by another thread $t_{u_2}$ $\rightarrow$ $t_l$ visits the data table location $e'$, seeing the effect of update $u_2$.\footnote{%
		It may also be that $t_{u_1} \equiv t_{u_2}$.
	}
\end{enumerate}

To implement a cost-efficient and consistent iteration seems to be possible, based on the following design.

\begin{enumerate}
	\item The iterating thread $t$ begins a Synchronous Operation (\S\ref{subsec:synchronous-operations});
	\item it generates an iteration ID $i$;\footnote{%
		Given that there will be need to store some data per iterator, it may be that the iteration ID is the pointer to this alleged C structure.
	}
	\item it allocates a new local pages array of the same size of the used portion of the data table's pages array;
	\item it traverses the \texttt{AtomicDict}'s pages array, and for each page:
	\begin{enumerate}
		\item it sets the \texttt{iteration} field on the page to $i$;
		\item it copies the page pointer into the local pages array;\footnote{%
			This copy can actually be done separately using \texttt{memcpy} on the pages array directly, which should significantly improve performance, since \texttt{memcpy} is such a highly optimized kernel function.
		}
		\item it increments the page reference count;
	\end{enumerate}
	\item $t$ ends the Synchronous Operation; and
	\item proceeds iterating over the new pages.
	\item When the iteration completes, it decrements the reference count of the pages in the local pages array, and frees the array.
\end{enumerate}

When the iteration field of a page is set (i.e.\ $\neq$ \texttt{NULL}), the page is considered to be immutable.
A thread that wishes to mutate it, should thus first copy the page's contents into a newly allocated page, and then atomically swap the two pages in the pages array.
If the swap fails, then it means that another thread has swapped the page into a mutable page: it cannot be that the newly swapped-in page is immutable because that requires holding all of the accessors' locks.

The creation of new mutable pages is therefore entirely offloaded from the iterating thread, to the other \texttt{AtomicDict}'s accessors.
This is a feature of the design.
Consider the opposite case in which the iterating thread additionally has to allocate more pages and swap them in the pages array, or equivalently has to create its own immutable pages.
In this scenario the Synchronous Operation itself would take much longer.
Thus, hindering the performance of the other accessors which are trying to apply mutations to the hash table (recall that lookups are permitted in spite of Synchronous Operations).
That is, there is a missed opportunity cost associated with it.

Instead, the parallelization of threads which independently swap the immutable pages out for their mutable copies, entails that multiple threads can be used to create the new pages.
And, furthermore, that if no mutations were actually going to happen, either at all or in a subset of the pages, then strictly less work is required overall.

It may also be entirely possible to permit the re-usage of the iteration ID\@.
So that if a thread sees that all pages refer to the same iteration ID, then it avoids marking the pages as immutable, and instead copies the associated data into its own iterator.


\paragraph{Linearization and non-lock-freedom.}
With this being a Synchronous Operation, it is trivially not lock-free.
Its linearization also follows the semantics of the Synchronous Operation: the iterator linearizes when it manages to acquire all the required locks.
