%%%%%%%%%%%%%%%%%%%%%%%%%%%%%%%%%%%%%%%%%%%%%%%
%
% Template per Elaborato di Laurea
% DISI - Dipartimento di Ingegneria e Scienza dell’Informazione
%
% update 2015-09-10
%
% Per la generazione corretta del 
% pdflatex nome_file.tex
% bibtex nome_file.aux
% pdflatex nome_file.tex
% pdflatex nome_file.tex
%
%%%%%%%%%%%%%%%%%%%%%%%%%%%%%%%%%%%%%%%%%%%%%%%

% formato FRONTE RETRO
\documentclass[a4paper]{book}
\usepackage{epsfig}
\usepackage{lipsum}
\usepackage{graphicx}
\usepackage{pgffor}
\usepackage{epigraph}
\usepackage[hidelinks]{hyperref}

% using XeLaTeX

\usepackage[english]{babel}
\usepackage{csquotes}

\def\mtoday{\the\year-\twodigits\month-\twodigits\day}
\def\twodigits#1{\ifnum#1<10 0\fi\the#1}

\usepackage[backend=biber, dateabbrev=false, urldate=comp]{biblatex}
\addbibresource{biblio.bib}

\usepackage{xcolor}
\usepackage{listings}
\usepackage{amsmath}
\usepackage{amsthm}
\usepackage{mathtools}

\newtheorem{proposition}{Proposition}
\newtheorem{lemma}{Lemma}
\newtheorem{corollary}{Corollary}
\theoremstyle{definition}
\newtheorem{definition}{Definition}
\theoremstyle{remark}
\newtheorem{axiom}{Assumption}
\newtheorem{observation}{Observation}
\newtheorem{example}{Example}

\newcommand{\source}[1]{\ensuremath{\text{source}\left[{#1}\right]}}
\newcommand{\dest}[1]{\ensuremath{\text{dest}\left[{#1}\right]}}
\newcommand{\isfinal}[1]{\ensuremath{\text{isFinal}\left({#1}\right)}}

\newcommand{\ind}{\ensuremath{\phantom{= =}}}

\usepackage[useregional]{datetime2}
\DTMsavedate{publish day}{2021-02-10}
\newcommand{\lastvisited}{, last visited \DTMusedate{publish day}.}


\begin{document}

    % nessuna numerazione
    \pagenumbering{gobble} 
    \thispagestyle{empty}

\begin{center}
  \begin{figure}[h!]
    \centering{\psfig{file=marchio_unitrento_colore_it_202002.eps, width=0.6\textwidth}}
  \end{figure}
  

  \vspace{1 cm} 

  \LARGE{Department of Information Engineering and Computer Science\\}

  \vspace{1 cm} 
  \Large{Master's Degree in Computer Science
  }

  \vspace{2 cm} 
  \Large\textsc{Final Dissertation\\} 
  \vspace{1 cm} 
  \huge\textsc{A State of the Art Concurrent Lock-Free Hash Table\\}
  \Large{\it{With an implementation for the CPython Interpreter}}


  \vspace{2 cm} 
  \begin{tabular*}{\textwidth}{ c @{\extracolsep{\fill}} c }
  \Large{Supervisor} & \Large{Student}\\
  \Large{Prof. Paola Quaglia}& \Large{Daniele Parmeggiani}\\
  \end{tabular*}

  \vspace{2 cm} 

  \Large{Academic year 2023/2024}
  
\end{center}

    \cleardoublepage
 
%%%%%%%%%%%%%%%%%%%%%%%%%%%%%%%%%%%%%%%%%%%%%%%%%%%%%%%%%%%%%%%%%%%%%%%%%%
%%%%%%%%%%%%%%%%%%%%%%%%%%%%%%%%%%%%%%%%%%%%%%%%%%%%%%%%%%%%%%%%%%%%%%%%%%
%% Nota
%%%%%%%%%%%%%%%%%%%%%%%%%%%%%%%%%%%%%%%%%%%%%%%%%%%%%%%%%%%%%%%%%%%%%%%%%%
%% Sezione Ringraziamenti opzionale
%%%%%%%%%%%%%%%%%%%%%%%%%%%%%%%%%%%%%%%%%%%%%%%%%%%%%%%%%%%%%%%%%%%%%%%%%%
%%%%%%%%%%%%%%%%%%%%%%%%%%%%%%%%%%%%%%%%%%%%%%%%%%%%%%%%%%%%%%%%%%%%%%%%%%
    %\begin{center}
		%\emph{
		%	
		%}
    %\end{center}
    %\clearpage
%  \pagestyle{plain} % nessuna intestazione e pie pagina con numero al centro

  
    % inizio numerazione pagine in numeri arabi
    \mainmatter

%%%%%%%%%%%%%%%%%%%%%%%%%%%%%%%%%%%%%%%%%%%%%%%%%%%%%%%%%%%%%%%%%%%%%%%%%%
%%%%%%%%%%%%%%%%%%%%%%%%%%%%%%%%%%%%%%%%%%%%%%%%%%%%%%%%%%%%%%%%%%%%%%%%%%
%% Nota
%%%%%%%%%%%%%%%%%%%%%%%%%%%%%%%%%%%%%%%%%%%%%%%%%%%%%%%%%%%%%%%%%%%%%%%%%%
%% Si ricorda che il numero massimo di facciate e' 30.
%% Nel conteggio delle facciate sono incluse 
%%   indice
%%   sommario
%%   capitoli
%% Dal conteggio delle facciate sono escluse
%%   frontespizio
%%   ringraziamenti
%%   allegati    
%%%%%%%%%%%%%%%%%%%%%%%%%%%%%%%%%%%%%%%%%%%%%%%%%%%%%%%%%%%%%%%%%%%%%%%%%%
%%%%%%%%%%%%%%%%%%%%%%%%%%%%%%%%%%%%%%%%%%%%%%%%%%%%%%%%%%%%%%%%%%%%%%%%%%

    % sommario
    \chapter*{Abstract}
\label{ch:abstract}

\addcontentsline{toc}{chapter}{Abstract}

The Python community is increasingly interested in parallelization.
One of the biggest changes to the CPython interpreter, the most common implementation of the language, is set to be released this October 2024.
In it, the greatest long-standing bottleneck to multithreading performance scaling is removed: the Global Interpreter Lock.
We can expect more Python programmers to start choosing multithreading concurrency models in the future.
We can also expect concurrent data structures to gather increasing interest.

Concurrent hash tables among the most important concurrent data structures, and the built-in sequential hash table of CPython is ubiquitously used within the interpreter and throughout the community.
We find that the research for efficient, concurrent hash tables has been very active in recent years, and many implementations are available for C programs.

We propose here an efficient concurrent hash table that can be used from Python code directly, albeit implemented in C.
We additionally introduced it to the Python community.

    
   	% indice
    \tableofcontents
    \clearpage
%%%%%%%%%%%%%%%%%%%%%%%%%%%%%%%%%%%%%%%%%%%%%%%%%%%%%%%%%%%%%%%%%%%%%%%%%%
%%%%%%%%%%%%%%%%%%%%%%%%%%%%%%%%%%%%%%%%%%%%%%%%%%%%%%%%%%%%%%%%%%%%%%%%%%
%% Nota
%%%%%%%%%%%%%%%%%%%%%%%%%%%%%%%%%%%%%%%%%%%%%%%%%%%%%%%%%%%%%%%%%%%%%%%%%%
%% Sommario e' un breve riassunto del lavoro svolto dove si descrive 
%% l’obiettivo, l’oggetto della tesi, le metodologie e 
%% le tecniche usate, i dati elaborati e la spiegazione delle conclusioni 
%% alle quali siete arrivati.
%% Il sommario dell’elaborato consiste al massimo di 3 pagine e deve contenere le seguenti informazioni: 
%%   contesto e motivazioni
%%   breve riassunto del problema affrontato
%%   tecniche utilizzate e/o sviluppate
%%   risultati raggiunti, sottolineando il contributo personale del laureando/a
%%%%%%%%%%%%%%%%%%%%%%%%%%%%%%%%%%%%%%%%%%%%%%%%%%%%%%%%%%%%%%%%%%%%%%%%%%
%%%%%%%%%%%%%%%%%%%%%%%%%%%%%%%%%%%%%%%%%%%%%%%%%%%%%%%%%%%%%%%%%%%%%%%%%%      
      
    %%%%%%%%%%%%%%%%%%%%%%%%%%%%%%%%
    % lista dei capitoli
    %
    % \input oppure \include
    %
    \chapter{Introduction}

\section{Hash tables, concurrent hash tables, and their applications}

\cite{art-mp}

\section{Python's builtin hash table}

\cite{hettinger-dict}
\cite{dict-notes}
\cite{dict-comment-design}
\cite{dict-comment-hash}

\section{Free-threading Python}

An account of the free-threading changes.

\cite{dabeaz-gil}
\cite{gross-doc}
\cite{mimalloc}

\subsection{Concurrent, biased reference counting}

\cite{biased-refcounting}

\subsection{Deferred reference counting}

\cite{deferred-refcounting}

\subsection{Quiescent State-Based Reclamation}

\cite{qsbr}

\subsection{Garbage collection}

\cite{pep703}

\subsection{Thread-safety of builtin data structures}

\cite[\S Container Thread-Safety]{pep703}

    \chapter{Review of the State of the Art}

\section{Maier's concurrent hash table}

\cite{maier}

\section{Bolt}

\cite{bolt}

\section{Python's sequential hash table}

\subsection{With nogil/free-threading changes}

\section{DRAMHiT}

\cite{dramhit}

\section{On the feasibility of completely lock-free hash tables}

\cite{hesselink}

    \chapter{Proposed Design and Implementation}\label{ch:design-and-implementation}

The implementation is available online at~\cite{cereggii}.

\cite{peniocereus-greggii}

$O(t)$

$\textrm{atw} \textrm{rgw} \textrm{rgr}$

\section{Design overview}\label{sec:design-overview}

\subsection{Python's dictionary}\label{subsec:python-dict}

A lot of the basis for \texttt{AtomicDict} has been inspired from Python's \texttt{dict}.
Namely, that the hash table itself is instead an index over another table called the data table.
In the latter is where the pointers to the keys and values objects reside.

This improves on both the storage space and the iteration speed.
In fact, if we call $F_r$ the fill ratio of the dictionary (i.e.\ the number of occupied slots over the total number of slots), there not need be \~$(1 - F_r)$ amount of empty 16-bytes entries, for the unused slots in the hash table.
Instead, there will be \~$1 - F_r$ empty $O(8)$-bytes slots in the index, while the keys and value pairs will be stored in a separate compact table.
That is, the data table will have a fill ratio of \~100\%.
Thus, when iterating over the keys in the dictionary, the compact data table will serve as the basis for the iteration, instead of the sparse index.
Also, note that the size of index nodes will vary depending on the size of the key set: when the key set is small there is no need to index the data table with 64-bit numbers, instead the size of nodes is chosen dynamically, from a lower bound of 1-byte, when the cardinality of the key set is $\leq 64$.
As the set of keys becomes larger than the maximum allowed fill ratio,\footnote{%
	I.e. $2/3$, both in \texttt{dict} and \texttt{AtomicDict}.
} then the migration process is initiated, which does not usually touch the data table, but only the smaller index instead.
The data table is only modified when it needs to be reduced after a large number of deletes, which are supposedly modest in number.

As described in~\cite[\S13.1]{art-mp}, we should be inclined to think that the usage of the dictionary consists of 90\% lookups, 9\% insertions, and 1\% deletes.
In \S\ref{sec:dict-metrics} we confirm this rule-of-thumb.
We further expose a counter-intuitive notion: most lookups fail.
That is, most lookups return that the key searched for, was not found in the hash table.\footnote{%
	See \S\ref{sec:dict-metrics} on \texttt{\_Py\_dict\_lookup} for the full details.
}

This may be explained by the presence of iterations, that is, of operations that return the entire currently stored set of keys.
Thus, it may be that in most cases, when a program looks up one specific key in the dictionary, it really isn't known whether the key is there or not.
And furthermore, instead of looking up individual keys which are known to be there, programs prefer to iterate over all keys.
This would result in individual lookups having a tendency to fail.

Based on this, we enhance the table hashing with the Robin Hood scheme, known in the literature~\cite{robin-hood,bolt}, which is especially effective in pruning searches for keys which are not in the table.
Our approach, detailed in the following \S\ref{subsec:lazy-robin-hood}, differs from~\cite{bolt} in that we don't necessarily constrain the dictionary to this scheme: we maintain the Robin Hood invariants, described later, as long as we can do so in a lock-free manner.
When the maintenance of those invariants requires an increasing number of atomic operations to be carried out in the index, we instead fall back to regular linear probing.
When doing so, before actually inserting a key that does not maintain the Robin Hood invariants, we mark the hash table as not anymore compact.
A compact operation can be requested by the user to recover the Robin Hood invariants.
This operation essentially consists of enlarging the hash table so that the invariants can be maintained.

We also optionally store tags in the nodes, further stealing some bits, in order to reduce the necessity of lookups into the data table.
The tag contains a portion of the hash of the key in the data table, when the tag does is not the same as the equivalent portion of the hash of the key looked up, the relevant entry in the data table is not visited.
This reduces the cost of having split index and data tables.


\subsection{Maier's}\label{subsec:maier}

Maier's hashmap implementation has an important difference when compared with Python's dictionary.
In the latter, the least-significant bits of the hash are used to determine the position in the linear array; that is, $d_0(x) = x \mod 2^s$, with $s$ being the logarithm base-2 of the dictionary size.
While in the former, the most-significant bits are used: $d_0 = x \gg (64 - s)$, where $\gg$ is a right-shift operation.

An extensive reasoning as to why Python's hashmap chose this scheme can be found in~\cite{dict-comment-hash}.
While the reason why Maier chose the right-shift based scheme is so that the relative position of two keys which are in the same probe doesn't change as much as in the alternative, when the array is resized.
That is, consider a key $x$ s.t. $h(x) \mathbin{\&} 2^{s+1} = 0$, and a key $y$ s.t. $h(x) \mathbin{\&} 2^{s+1} = 1$.
If they were colliding before resizing from $s$ to $s+1$(i.e. $h(x) \mathbin{\&} 2^s-1 = h(y) \mathbin{\&} 2^s-1$), they will not collide in the corresponding hashmap of size $s+1$.
But this creates an unexpected problem.
It is trivial to see that $d_0(x)' = d_0(x)$.
The $d_0$ position of $y$ will have changed from $d_0(y) = h(y) \mathbin{\&} 2^s-1$ to $d_0(y)' = h(y) \mathbin{\&} 2^{s+1}-1$.
In other words, $d_0(y)' = d_0(y) + 2^s$.

The problem with this behavior is that the position in the new array cannot be safely determined by looking at the prior array, without reading the actual hashes of each key in a probe.

This problem is avoided by choosing the most-significant bits scheme: $d_0(y) = h(y) \gg (64 - s)$, and $d_0(y)' = h(y) \gg (64 - s - 1)$.
In other words, $d_0(y)' = d_0(y) \cdot 2 + 1$.
While $d_0(x) = h(x) \gg (64 - s)$, and $d_0(x)' = h(x) \gg (64 - s - 1) \Rightarrow d_0(x)' = d_0(x) \cdot 2$.

It follows that the nodes of a cluster in the prior array will ``stay close to each other'' in the new array.
This behavior enables avoiding the necessity to exercise care in the moving of nodes from one generation of the hashmap to the next, resulting in much less synchronization overhead.

Though, given this fundamental difference in picking the hash bits that determine the position in the hashmap, it was necessary to re-hash Python-generated hashes, or else the $d_0$ position of very many objects would be $0$.
This is done with a cheap CRC32 hardware instruction, that Maier also used in his implementation.
Thus, the required property of hashes being ``\emph{large} pseudo-random numbers,'' is respected.

\subsection{Reservation Buffers}\label{subsec:reservation-buffers}

The usage of pages for the data table, implicitly creates zones of contention.
When threads want to add a new key into the hash table, the most-recently added page is the one in which the insertion is attempted.
So that effectively all threads are trying to contend one page of the data table.

The degree to which contention is exhibited depends on the strategy with which threads decide which free entry in the page they wish to take for the new item.
Consider a strategy in which the lowermost index is always chosen: every thread always tries to reserve the same entry, with only one thread succeeding at any given time.

A simple, yet much better, alternative is to treat the page itself as a sort of hash table.
That is, instead of choosing an entry based on the current availability, an entry is chosen based on the hash of the key.
Thus, with sufficiently uniformly distributed hash values, the contention is greatly reduced.

The reservation itself needs to be carried out with atomic operations, so that the cost of inserting one key is always at least two atomic operations: one write to the data table, and one write to the index.
In order to amortize the cost of having a data table separate from the actual hash table (the index), instead of reserving one entry at every insertion, we let threads reserve a configurable number of entries, four per default.\footnote{%
	The allowed values are 1, 2, 4, 8, 16, 32, or 64.
	With 64 being the size of an entire data table page.
}
When a thread has no available reservations, it resolves to finding a free entry based on the key's hash as described before, reserving four entries; instead when a thread has reservations at disposal, it directly writes into the free entries that it owns, using regular (non-atomic) writes.

In summary, the expected number of atomic writes per inserted key is $1 + 1/4 = 1.25$.

\subsection{Accessor Storage}\label{subsec:accessor-storage}

The reservation buffer, along with other things, is stored inside an accessor-local storage.
It contains:
\begin{enumerate}
	\item a mutex;
	\item a local view of the size of the dictionary;
	\item a flag indicating whether this accessor (thread) is participating in the current migration; and
	\item the reservation buffer.
\end{enumerate}

The mutex itself protects the contents of the accessor storage.
This may seem counter-intuitive, but it is actually very useful.
First of all, when a dictionary needs to be freed, all the allocated accessor storages need to be freed as well.
In order to do this, the accessor storages are kept in a list.
A thread freeing the dictionary traverses the list to free the accessor storages, and a thread accessing the dictionary for the first time, appends its newly created accessor storage to the list.

The mutex itself is thus generally held only by its accessor, which releases it at the end of any insert, or delete operation.

Furthermore, when a thread becomes the leader of a migration (detailed later), it may necessitate to modify the reservation buffers of the other threads, in case the data table is modified.
E.g.\ during a shrink migration, the data table is shrunk, as well as the index, in order to free unoccupied space.
That also entails that the entries in the reservation buffers need to be changed, because their location relative to the start of the data table has changed.


\subsection{Synchronous Operations}\label{subsec:synchronous-operations}

In abstract terms, the presence of the list of thread-local mutexes enables the creation of cuts in the execution of dictionary operations, whereas the acquisition of all the thread-local mutexes by one thread creates a distinction between the operations that happened before this circumstance and those that happened after it.

This crucial characteristic, which is required e.g.\ for the re-ordering of data table entries, enables also many more usages.
A few are described in the following sections.
In particular, it enables a very simple mechanism for ensuring that all accessors come to know the presence of a hash table migration, for establishing the correct size of the hash table, and for performing a sequentially consistent iteration of the items in the hash table.

All of these operations are called synchronous because they all share the common necessity to be carried out sequentially by one thread, or rather that at least one part of their execution needs to be performed sequentially.
For instance, the hash table migration enjoys the help of more than one thread, but requires a step in which the leader performs the necessary alterations to the data table before other threads can join in the migration.

The addition of this mechanism ensures that the dictionary presented here is capable both of performant lock-free operations, and of complex operations that require the exclusive access to the entire dictionary in order to be performed without a prohibitive number of expensive atomic memory writes.
The properties that a synchronous operation is sequential and ensured to be mutually exclusive w.r.t.\ all other threads, make it also very simple to be explained and understood, an important characteristic for collaborative development.


\subsection{Lazy Robin-Hood}\label{subsec:lazy-robin-hood}

The Robin Hood state for the hash table is essentially the distance each key has from its distance-0 position.
In other words, the number of collisions.

In order to keep track of this distance, we steal some bits from the nodes in the index.
According to~\cite[Corollary to Theorem~3]{robin-hood}, the expected number of collisions per probe for a full table is $\Theta(\log n)$, e.g.\ four for a table of size 64.
Such is a relatively low number, thus we can steal two bits, or in general $\log \log n$ bits, from the index nodes.

We do so, by following the values in Table~\ref{tab:robin-hood-nodes}.
As you can observe the proposed hash table has definite minimum and maximum sizes, respectively of $2^{6} = 64$ and $2^{56} \approxeq 7.2 \times 10^{16}$.

\begin{table}
	\centering\begin{tabular}{llll|llll}
		$\log n$ & Node & Distance & Tag & $\log n$ & Node & Distance & Tag\\
		\hline
		6 & 8 & 2 & 0 & 32 & 64 & 5 & 27 \\
		7 & 16 & 3 & 6 & 33 & 64 & 6 & 25 \\
		8 & 16 & 3 & 5 & 34 & 64 & 6 & 24 \\
		9 & 16 & 4 & 3 & 35 & 64 & 6 & 23 \\
		10 & 16 & 4 & 2 & 36 & 64 & 6 & 22 \\
		11 & 32 & 4 & 17 & 37 & 64 & 6 & 21 \\
		12 & 32 & 4 & 16 & 38 & 64 & 6 & 20 \\
		13 & 32 & 4 & 15 & 39 & 64 & 6 & 19 \\
		14 & 32 & 4 & 14 & 40 & 64 & 6 & 18 \\
		15 & 32 & 4 & 13 & 41 & 64 & 6 & 17 \\
		16 & 32 & 4 & 12 & 42 & 64 & 6 & 16 \\
		17 & 32 & 5 & 10 & 43 & 64 & 6 & 15 \\
		18 & 32 & 5 & 9 & 44 & 64 & 6 & 14 \\
		19 & 32 & 5 & 8 & 45 & 64 & 6 & 13 \\
		20 & 32 & 5 & 7 & 46 & 64 & 6 & 12 \\
		21 & 32 & 5 & 6 & 47 & 64 & 6 & 11 \\
		22 & 32 & 5 & 5 & 48 & 64 & 6 & 10 \\
		23 & 32 & 5 & 4 & 49 & 64 & 6 & 9 \\
		24 & 32 & 5 & 3 & 50 & 64 & 6 & 8 \\
		25 & 32 & 5 & 2 & 51 & 64 & 6 & 7 \\
		26 & 64 & 5 & 33 & 52 & 64 & 6 & 6 \\
		27 & 64 & 5 & 32 & 53 & 64 & 6 & 5 \\
		28 & 64 & 5 & 31 & 54 & 64 & 6 & 4 \\
		29 & 64 & 5 & 30 & 55 & 64 & 6 & 3 \\
		30 & 64 & 5 & 29 & 56 & 64 & 6 & 2 \\
		31 & 64 & 5 & 28 & \\
	\end{tabular}
	\caption{Sizes of internal fields of nodes in the index hash table. Let $n$ be the capacity of the table, and \emph{Node}, \emph{Distance}, and \emph{Tag} refer to their sizes, expressed in number of bits.}
	\label{tab:robin-hood-nodes}
\end{table}

Consider the general situation in which a new item is being inserted.
Paraphrasing from~\cite{robin-hood}:
\begin{quote}
	Suppose that element $A$ is at location $l$, at distance $d_a$ from its distance-0 slot, and $B$ is to be inserted (or moved).
	Suppose $B$ has been denied allocation of its first $d_b - 1$ choices and we are now considering $l$, its $d_b$th choice.
\end{quote}

Then, the Robin Hood invariant, or heuristic, maintains that:\footnote{
	As put by Celis et al. in~\cite{robin-hood}:
	\begin{quote}
		This procedure of ``taking from the rich and giving to the poor'' gives rise to the name Robin Hood hashing.
		Like Robin Hood, it does not change the average wealth (mean probe length), only its distribution.
		It is clear that the principal effect is to substantially reduce the variance.
	\end{quote}
}

\begin{tabular}{rp{8cm}}
	if $d_a \geq d_b$: & $A$ retains $l$, the insertion of $B$ continues by considering its $(d_b + 1)$st choice; \\
	if $d_b > d_a$: & $B$ displaces $A$ from $l$, the insertion procedure for $A$ is applied starting with its next $(d_a + 1)$st choice; \\
\end{tabular}

Comparing with~\cite{robin-hood}, we have slightly modified the invariant so as to favor avoiding moving elements on a tie.

The laziness in our approach is that if the \emph{aligned} size of the write required to safely insert the node into the index exceeds 128-bits, then we refrain from maintaining the Robin Hood invariants at all, for we would need to employ more than one atomic writes in order to store the newly inserted key.
This is because we cannot use the hardware~\cite[CMPXCHG--Compare and Exchange]{x86-64} instruction over more than 16-bytes.
The instruction further requires that the write be 16-byte aligned, or else a segmentation fault is encountered.\footnote{%
	Other faults may be encountered as well, depending on the architecture, for instance in aarch64 ISAs (which are not currently supported; see~\S\ref{subsec:compatibility-with-other-isas}) a bus fault is encountered.
}
Thus, e.g.\ if a new node is inserted into the index after a 16-byte boundary, the insertion cannot be performed with a normal (i.e.\ compact) node.

That is, when the distance cannot be stored in $O(\log \log n)$ space, we semantically say the distance is $\infty$ and store it as $\log \log n - 1$ (we may later refer to this as the maximum distance for a given size of the table, or simply as the maximum distance).
When a node has distance equals to the maximum distance, we say that node is a non-compact node.
We also employ non-compact nodes to denote tombstones.
As is well known, see for instance~\cite[\S6.4]{the-art-vol-2}, ``the obvious way of deleting records from a hash table doesn't work.''
Instead of simply removing a node, we swap it for a tombstone node, which is a non-compact node pointing to entry number 0 of the data table.
The data table maintains the invariant that the key and value fields of that entry are always \texttt{NULL}.\footnote{%
	This is achieved by requiring at initialization time that the thread creating the \texttt{AtomicDict} instance reserves entry number 0 for itself, and marks it as already occupied, without writing anything into it.
	Thus, all data tables have at least one unused entry.
	Furthermore, if the thread that created the hash table never writes into it, there will always be $|R|$ unused entries, with $|R|$ being the size of the reservation buffer.
	For more details on reservations see \S\ref{subsec:reservation-buffers}.
}
Operations that read the index, simply always skip non-compact nodes that point to entry 0.
Further details on tombstone nodes and deletes are given in \S\ref{sec:delete}.

When a new node could be inserted, respecting the requirements for our lazy robin hood, but it should be inserted at a distance greater then the maximum, we initiate a grow migration, as described in~\S\ref{sec:migrations}.
In effect, we have a local trigger for our migrations that doesn't require global knowledge of the table.
We can observe the local effect of collisions to determine that the table size is too small, and take action accordingly.
This is not the only trigger for migrations, for a complete list refer to the relevant section.
One may argue that a degenerate set of hashes can continuously trigger migrations.
While this is true in principle, recall that we essentially make use of double-hashing (see above~\S\ref{subsec:maier}): we use the Python hash to compute the CRC32 hash, making such a set of keys very hard to find.
In the future, we may additionally provide a configurable setting that inhibits this behavior and makes the insertion fallback to non-compact node insertion.


\section{Lookup}\label{sec:lookup}

\lstinputlisting[
	numbers=left,
	stepnumber=5,
	numberfirstline=true,
	language={C},
	caption={Lookup Operation},
	label={lst:lookup}
]{listings/lookup.c}

\subsection{Linearization}
\subsection{Lock-freedom}

\section{Insert or update}\label{sec:insert-or-update}

\lstinputlisting[
	numbers=left,
	stepnumber=5,
	numberfirstline=true,
	language={C},
	caption={ExpectedInsertOrUpdate Operation},
	label={lst:insert}
]{listings/insert.c}

The InsertOrUpdate (or rather, ExpectedInsertOrUpdate) routine departs from Maier's design in that it doesn't expect the user to providde an update function, but rather mimics more the compare and set routines of other atomic data structures: it takes as input the expected and desired values, and it returns the value that was stored before the update was performed (or a special NOT\_FOUND object for a successful insertion, in which case either NOT\_FOUND ot ANY were the expected value.)
If the expected value was not the stored value (or the compare and set low-level call failed and the newly read value then differed from the expected one), then a special EXPECTATION\_FAILED object is returned.

There's also another special value called ANY that can be used as the expected value to signify that whatever the current value is, even an absent value, it should be replaced with the desired value.
The three special objects ANY, NOT\_FOUND, and EXPECTATION\_FAILED cannot be used as keys or values in AtomicDict, so as to guarantee their semantics are consistent (cf.\ EXPECTATION\_FAILED being a value returned by this routine.)

If the expected value is either ANY or NOT\_FOUND, there is an insertion fast-path used in which the $d_0$ element in the index is CASed with the entry reserved without going through the currently stored data in the index.
If this CAS succeeds, then the operation is completed.
If it doesn't the general code path is followed.
Notice that this doesn't compromise correctness: if the $d_0$ node was in fact empty, it means that the key wasn't in the dictionary.

\subsection{Linearization}
\subsection{Lock-freedom}

\section{Delete}\label{sec:delete}

\lstinputlisting[
	numbers=left,
	stepnumber=5,
	numberfirstline=true,
	language={C},
	caption={Delete Operation},
	label={lst:delete}
]{listings/delete.c}

\subsection{Linearization}
\subsection{Lock-freedom}

\section{Migrations: growth, shrinking, and compaction}\label{sec:migrations}

\subsection{Linearization}
\subsection{Non-lock-freedom}

\section{Approximate size}\label{sec:approximate-size}

\subsection{Linearization}
\subsection{Lock-freedom}

\section{Consistent size}\label{sec:consistent-size}

\subsection{Linearization}
\subsection{Non-lock-freedom}

\section{Batch lookup}\label{sec:batch-lookup}

Inspired by the design of DRAM-HiT, this method was added so as to amortize the cost of multiple memory accesses when a program wants to read a number of keys from the hash table.
It deviates from~\cite{dramhit} in that it does not require the user to continuously poll the table for results.
Instead, a batch of keys can be submitted for lookup, in one invocation.

An example call:

\begin{lstlisting}[language=Python]
foo = AtomicDict({'a': 1, 'b': 2, 'c': 3})
result = foo.batch_getitem({
  'a': None,
  'b': None,
  'f': None,
})
assert result == {
  'a': 1,
  'b': 2,
  'f': cereggii.NOT_FOUND,
}
\end{lstlisting}

The \texttt{<cereggii.NOT\_FOUND>} object is a special, global object that cannot be inserted into \texttt{AtomicDict}.

The operation involves the prefetching of each distance-0 location of the keys in the batch.
This results in an amortization of the cost of individual memory accesses: if the number of prefetches issued is sufficiently large (how large depends on the hardware), then the processor will be stalled waiting for the first read, while the subsequent ones will be read from the cache without incurring in the cost of main memory access.
This behavior is cheaper-by-comparison: it is cheaper to prefetch several keys, possibly blocking at the first one read, than to block at every read.

NUMA architectures will enjoy this mechanism particularly: when prefetches are issued at remote memory addresses, the comparative cost of the amortized reads is much less than the sum of the costs of each individual read.

In talking about NUMA architectures, it is generally also possible to argue that the cost of a remote access may be hidden by the possibly many local memory accesses that precede it.
Such is not our case: the index, which is allocated as a contiguous array, will reside entirely in one NUMA node.
Thus, a thread accessing the index will either only issue operations to remote memory, or to local memory.\footnote{%
	A further endeavor may comprise the partitioning of a hash table into several shards, which in turn are managed by one or multiple threads associated to a single NUMA node, such that any given key may only reside in one shard (e.g. by using the least-significant bits of the hash to associate a key with a shard).
	
	Mimicking the design of DRAM-HiT, the user thread would submit requests to a message queue which is consumed by the threads associated with a shard.
	The design need not be asynchronous as in \cite{dramhit} to gain the speed-up of stronger memory associativity. XXX check this
}

The batch is subdivided into chunks of a configurable size.
This is intended to prevent the over-fetching of memory.
That is, the cache of an individual processor may become overly filled with the prefetched portions of the index, and the hardware will proceed to evict lines of cache before they are actually read by the lookups that requested them.

\subsection{Linearization}

The linearization of this method is intended as the individual linearizations of the distinct lookup operations.
Each distinct lookup is linearizable, although the overall invocation is not.
Therefore, it is possible for a result of this method to not be sequentially consistent.


\subsection{Lock-freedom}

\section{Aggregate}\label{sec:aggregate}

\subsection{Reduce}\label{subsec:reduce}

\subsection{Linearization}
\subsection{Lock-freedom}

\section{Iterations}\label{sec:partitioned-iterations}

\subsection{Iteration Partitioning}\label{subsec:iteration-partitioning}

\subsection{Linearization}
\subsection{Lock-freedom}

\section{Proposal for a consistent iteration}\label{sec:consistent-iteration}

When a thread iterates over the hash table with the iterator described above in \S\ref{sec:partitioned-iterations}, it may encounter sequentially inconsistent behaviors, if other threads are concurrently mutating the table.
For instance:
\begin{enumerate}
	\item the same key is emitted more than once; or
	\item an update $u_1$ that happened strictly before another $u_2$ is not seen, while the latter is.\footnote{%
		Such is the case when two updates are executed in succession by the same thread.
		Thus, it is definitely known that $u_1 \rightarrow u_2$.
	}
\end{enumerate}

These may be very surprising behaviors for a programmer trying to debug an incorrect execution of an iterating thread.
The necessity for a sequentially consistent iterator lies therein.

\subsection{Linearization}
\subsection{Non-lock-freedom}

    \chapter{Measurements}

\section{Metrics on usage of Python's builtin hash table}\label{sec:dict-metrics}

\cite{dict-metrics}

\section{Performance of hardware atomic operations}

\cite{cas-costs}

\section{Comparison with other reviewed hash tables}

    \chapter{Conclusions}\label{ch:conclusions}

\section{Future Work}\label{sec:future-work}

Apart from the work already described in Chapter~\cite{ch:design-and-implementation}.

\subsection{Transactions}\label{subsec:transactions}

\subsection{Compatibility with Python 3.13 (free-threading)}\label{subsec:compatibility-with-3.13-free-threading}

\texttt{Py\_mod\_gil}

\subsection{Compatibility with Python 3.13 (default build)}\label{subsec:compatibility-with-3.13-default-build}

\subsection{Compatibility with other ISAs}\label{subsec:compatibility-with-other-isas}

\subsection{Compatibility with Older Versions of Python}\label{subsec:compatibility-with-older-versions-of-python}

\subsection{API Compatibility with built-in \texttt{dict}}\label{subsec:api-compatibility-with-dict}

\subsection{Compatibility with Sub-Interpreters}\label{subsec:compatibility-with-sub-interpreters}

\subsection{\texttt{AtomicDictHandle}}\label{subsec:atomicdicthandle}

\subsection{\texttt{AtomicCounter}}\label{subsec:atomiccounter}
\S\ref{sec:atomicshardedintcounter}
API-compatible with \texttt{collections.Counter}

      
	\sloppy
    \printbibliography\addcontentsline{toc}{chapter}{Bibliography}
    \fussy

%%%%%%%%%%%%%%%%%%%%%%%%%%%%%%%%%%%%%%%%%%%%%%%%%%%%%%%%%%%%%%%%%%%%%%%%%%
%%%%%%%%%%%%%%%%%%%%%%%%%%%%%%%%%%%%%%%%%%%%%%%%%%%%%%%%%%%%%%%%%%%%%%%%%%
%% Nota
%%%%%%%%%%%%%%%%%%%%%%%%%%%%%%%%%%%%%%%%%%%%%%%%%%%%%%%%%%%%%%%%%%%%%%%%%%
%% Nella bibliografia devono essere riportati tutte le fonti consultate 
%% per lo svolgimento della tesi. La bibliografia deve essere redatta 
%% in ordine alfabetico sul cognome del primo autore. 
%% 
%% La forma della citazione bibliografica va inserita secondo la fonte utilizzata:
%% 
%% LIBRI
%% Cognome e iniziale del nome autore/autori, la data di edizione, titolo, casa editrice, eventuale numero dell’edizione. 
%% 
%% ARTICOLI DI RIVISTA
%% Cognome e iniziale del nome autore/autori, titolo articolo, titolo rivista, volume, numero, numero di pagine.
%% 
%% ARTICOLI DI CONFERENZA
%% Cognome e iniziale del nome autore/autori (anno), titolo articolo, titolo conferenza, luogo della conferenza (città e paese), date della conferenza, numero di pagine. 
%% 
%% SITOGRAFIA
%% La sitografia contiene un elenco di indirizzi Web consultati e disposti in ordine alfabetico. 
%% E’ necessario:
%%   Copiare la URL (l’indirizzo web) specifica della pagina consultata
%%   Se disponibile, indicare il cognome e nome dell’autore, il titolo ed eventuale sottotitolo del testo
%%   Se disponibile, inserire la data di ultima consultazione della risorsa (gg/mm/aaaa).    
%%%%%%%%%%%%%%%%%%%%%%%%%%%%%%%%%%%%%%%%%%%%%%%%%%%%%%%%%%%%%%%%%%%%%%%%%%
%%%%%%%%%%%%%%%%%%%%%%%%%%%%%%%%%%%%%%%%%%%%%%%%%%%%%%%%%%%%%%%%%%%%%%%%%%

    \appendix
    \chapter{Atomic Reference}

\section{Design}

\section{Proof of correctness in nogil fork}

\section{Issues with free-threading changes and proposed re-implementation}

    \chapter{Atomic Integer}\label{ch:atomic-int}

This Appendix discusses the rationale and design of an atomic integer for CPython.
It can be useful both in the context of the free-threading changes, and in the default build of CPython.
The implementation discussed here is available in~\cite[src/cereggii/atomic\_int]{cereggii}.


\section[The necessity before and after the free-threading changes]{The necessity before and after the free-\\threading changes}\label{sec:the-necessity-before-and-after-the-free-threading-changes}

With the presence of the GIL, many concurrency problems that existed in Python code may never be observed.
An atomically updatable integer is a very simple abstraction to ensure that certain concurrency problems are handled correctly.

\begin{quote}
    [\ldots] There's currently no atomic int in Python and people go through all sorts of weird contortions to atomically count things (which often leads to broken code even with the GIL).\footnote{%
    This quote is from a comment by Gross in a public discussion, and is accessible at \url{https://discuss.python.org/t/towards-a-python-util-concurrent/41325/8}, last accessed \today.
}
\end{quote}

Regardless of the CPython build, a class of the likes of \texttt{AtomicInt} is still useful in Python's ecosystem.
It can additionally gain usefulness in the free-threading build because of the performance improvements that can be gained with specific design choices described later in Section~\ref{subsec:atomic-int-handles}.


\section{Design}\label{sec:atomic-int-design}

The design of \texttt{AtomicInt} revolves around exposing common C atomic operations on 64-bit signed integers to Python code.
Therefore, most of the implemented routines essentially create this mapping between C operations and Python APIs.

The APIs of \texttt{AtomicInt} closely resemble some of Java's \texttt{AtomicInteger} methods.
The \texttt{AtomicInt} has some easily expected methods, like \texttt{{get()}} and \texttt{{set()}}, and also some borrowed from Java, and very similar to other C-level atomic operations, such as \texttt{getAndSet()} and \texttt{{incrementAndGet()}}.

Especially useful are the so-called ``in-place'' Python methods, such as \texttt{my\-\_int += 1}, \texttt{my\_int *= 2}, etc.
These methods are very effective for atomic operations because, instead of requiring a separation of reads and writes in Python code, they describe the entire mutation that the program wants to see applied.
The operation can thus be efficiently retried within the same call to the relevant C routine, which performs the retry when the integer's value is contended, following the semantics of CAS atomic operations\@.

The program can thus avoid worrying about the correct usage of atomic CAS primitives, when using in-place methods.
Notice the similarity with \texttt{Atomic\-Dict}'s proposed \texttt{reduce} method, which permits to do essentially the same thing: fully describe a mutation, to be applied correctly by the internals of the routine.
Instead, a usage similar to the following would yield inconsistent results, assuming that the \texttt{thread} function is effectively the code run by multiple threads:
\begin{lstlisting}[label={lst:atomic-int-bad}, language=Python]
    i = AtomicInt()

    def thread():
      for x in range(100):
        i = i + 1
\end{lstlisting}

What happens in this case is that the value that \texttt{i} is holding, is first retrieved, the sum is performed, and then the result is stored in \texttt{i}, similarly to calling \texttt{{i.set(i.get() + 1)}}.
These three actions comprising a mutation that employs two distinct methods of \texttt{AtomicInt}, cannot be correctly executed without the protection of an external lock.
That is, the two methods being linearizable, does not entail that two successive calls to them are linearizable; in fact, the opposite is true.


\subsection{Handles}\label{subsec:atomic-int-handles}

It can easily be observed by profiling~\cite[examples/atomic\_int/counter.py]{cereggii} that the running time is vastly dominated ($\approx$99\%) by concurrent reference counting and other CPython thread-safety features, with the atomic increment being such a cheap operation in comparison.
Such is just a vast amount of overhead.

The reference count of \texttt{AtomicInt} is required to be modified only when the threads start using the object and then eventually releasing it, upon finishing their work.
The many other reference counting operations are just overhead.

Therefore, a simple but very effective change was implemented.
The \texttt{Atomic\-Int} object does not necessitate being referenced directly by the thread that wants to mutate it, instead it is also possible to access it indirectly through a handle.
An \texttt{AtomicIntHandle} is an object that internally keeps a pointer to an \texttt{AtomicInt} object.
Externally, it provides exactly the same APIs as \texttt{AtomicInt}, so that using or not the handle does not require substantial changes to a program's code.
If the handle is created by the only thread that uses it, as it is intended, then it enjoys the fast-path for concurrent reference counting implemented in free-threading Python.
(See Section~\ref{sec:concurrent-biased-reference-counting}.)

The same reference counting operations, thus always performed by the proxy object's owner thread, would be substantially cheaper.
So much so, that the observed running time of~\cite[examples/atomic\_int/counter.py]{cereggii} drops by a staggering $\approx$95\%.

During an exchange with Sam Gross, the main author of the free-threading changes, he had indeed confirmed that the added burden of concurrent reference counting here exhibited is likely to remain in the free-threading build of CPython, for the foreseeable future.
Furthermore, he was convinced that using this design of handle indirection can be effective in contrasting these performance bottlenecks.


\section{Sharded Counter}\label{sec:atomicshardedintcounter}

There can be a useful specialization of \texttt{AtomicInt}.
To further reduce contention, it is possible to restrict the usage of \texttt{AtomicInt} to that of a counter and then optimize it.

In fact, if the integer itself is used as a counter, a very common application, then it is possible for the $T$ threads accessing it to have $T$ distinct counters upon which to count in isolation with each other.
Since additions are commutative, the $T$ counters can be summed together lazily only when the program wants to read the value of the counter, say by a \texttt{{get()}} operation.
The result might then be cached to avoid the repetition of work when the value hasn't changed.

In principle, this can be achieved in a manner similar to \texttt{AtomicDict}'s scheme for consistent size retrieval, described in Section~\ref{sec:consistent-size}.
Each counter would be stored in a thread-local structure which also contains a lock.
Such lock would behave in the same manner as \texttt{AtomicDict}'s accessors locks.

The thread-local counters would need to be stored in distinct regions of memory, so as to ensure that they are not stored into the same cache line, exhibiting false sharing.
That is, the CPU cache coherency protocol would be obliged to keep the single cache line consistent among processors, resulting in no improvement, quite the opposite, over the current implementation of \texttt{AtomicInt}.


\end{document}
